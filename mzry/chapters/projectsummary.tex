% !Mode:: "TeX:UTF-8"

\chapter{系统概要设计}
\section{系统环境}
在开始介绍系统的大体结构之前,首先介绍一下本系统的开发环境:
\begin{enumerate}
	\item 操作系统:OS X Mountain Lion 10.9.1
	\item 部署环境:运行于Virtual Box下的最新版本Arch Linux
	\item 建模软件:Visual Paradigm for UML
	\item 编辑软件:Eclipse、WebStorm、Vim
	\item 版本控制:Git
	\item 持续集成:TeamCity\footnote{一款功能强大的持续集成(Continue Integration)工具,可以让团队快速实现持续继承:IDE工具集成、各种消息通知、各种报表、项目的管理、分布式的编译等等。}
\end{enumerate}

本系统涉及到了服务器端、浏览器端、评测器三个模块,它们所使用的编程语言也都不同。服务器端我们采用的是JDK7.0标准下的Java语言,项目通过maven来组织和管理。浏览器端核心框架为AngularJS和JQuery,前端UI采用Bootstrap3,项目通过Grunt来组织和管理,编程语言为Less和Coffeescript。评测器采用C++语言,使用32位Linux API实现。

\section{系统框架设计}
\subsection{系统总体技术架构}
为确保系统的互操作性、适用性及长期的扩充性,我们应以标准的、开放的要求进行架构。本文所设计的在线评测系统,是在优化、改造原有设计的基础上,借助于分布式的应用模型及先进的MVC体系结构实现的。在服务器端我们集中安置了系统中的数据库、程序和一些其它的组件,而在客户端我们只需要浏览器,同一数据源为用户提供数据查询服务,如此一来也就确保了数据的完整性与及时性。在很多情况下,用户的需要会随时间的推移而改变,因此在业务的处理逻辑出现变化的情况下,我们只需要在服务器端进行程序的修改,随后就可以重新进行发布,这样就方便了程序的研发及发布,也不会对用户产生影响。本系统总体结构如图\ref{Architecture}所示。
\pic[htbp]{系统总体技术架构}{}{Architecture}

客户端发起一个HTTP请求后,经过一系列中间处理最终被分配到该连接对应的控制器上(Controller)。与标准MVC模型不同的是,我们通过一个叫做服务(Service)的中间件来和模型(Model)进行交互,服务调用Hibernate框架的数据访问对象(DAO)来进行数据的持久化操作。在一系列逻辑操作之后,控制器(Controller)根据结果来选择合适的视图(View)返回给客户端。

评测器模块作为一个独立的模块存在于web框架之外,它通过服务(Service)来查找等待评测的任务队列、进行评测和更新任务队列。

\subsection{系统模块结构}
根据图\ref{Architecture}的结构图,服务器端的模块可以进一步细化为如下结构:
\begin{itemize}
	\item 配置模块(config):负责项目的整体配置。
	\item 数据库模块(db):使用Hibernate框架来完成持久化操作。
	\item 评测器模块(judge):负责评测任务的调度和执行。
	\item 服务模块(service):充当控制器(Controller)和模型(Model)的桥梁。
	\item 实用工具(util):提供许多有价值的API。
	\item 网站模块(web):包含了控制器(Controller)和许多与服务无关的组件。
\end{itemize}

\subsubsection{配置模块}
该模块承担的任务是对服务器的基本运行参数进行配置,例如Spring MVC框架的初始化配置、Hibernate框架的属性配置等。

\subsubsection{数据库模块}
该模块中包含了数据访问对象(DAO)以及相关的类,例如用于和数据库进行映射操作的实体类(Entity),可以转换为HQL语言\footnote{HQL是Hibernate Query Language的简写,即 hibernate 查询语言:HQL采用面向对象的查询方式。}查询条件的条件类(Condition)和用于将模型与控制器隔离开来而用到的数据传输对象(DTO)。

\subsubsection{评测器模块}
该模块包含了一个评测器服务(Judge Service),它产生许多个评测线程来进行多线程评测。

\subsubsection{服务模块}
这个模块的主要作用是提供丰富的模型操作API,例如模型实例的新建、修改、查找等操作。

\subsubsection{实用工具}
这个模块包含了所有供内部使用的公共API。

\subsubsection{网站模块}
该模块最主要的部分是控制器(Controller)模块,控制器负责处理来自客户端的HTTP请求,并通过一定的逻辑选择合适的服务(Service)来完成用户的请求,同时根据情况将合适的视图(View)返回给用户。