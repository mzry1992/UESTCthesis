% !Mode:: "TeX:UTF-8"

\chapter{系统概要设计}
\section{系统环境}
在开始介绍系统的大体结构之前,首先介绍一下本系统的开发环境:
\begin{enumerate}
	\item 操作系统:OS X Mountain Lion 10.9.1
	\item 部署环境:运行于Virtual Box下的最新版本Arch Linux
	\item 建模软件:Visual Paradigm for UML
	\item 编辑软件:Eclipse、WebStorm、Vim
	\item 版本控制:Git
	\item 持续集成:TeamCity\footnote{一款功能强大的持续集成(Continue Integration)工具,可以让团队快速实现持续继承:IDE工具集成、各种消息通知、各种报表、项目的管理、分布式的编译等等。}
\end{enumerate}

本系统涉及到了服务器端、浏览器端、评测器三个模块,它们所使用的编程语言也都不同。服务器端我们采用的是JDK7.0标准下的Java语言,项目通过maven来组织和管理。浏览器端核心框架为AngularJS和JQuery,前端UI采用Bootstrap3,项目通过Grunt来组织和管理,编程语言为Less\footnote{LESSCSS是一种动态样式语言,属于CSS预处理语言的一种,它使用类似CSS的语法,为CSS的赋予了动态语言的特性,如变量、继承、运算、函数等,更方便CSS的编写和维护。}和Coffeescript\footnote{CoffeeScript是一套JavaScript的转译语言。受到Ruby、Python与Haskell等语言的启发,CoffeeScript增强了JavaScript的简洁性与可读性。}。评测器采用C++语言,使用32位Linux API实现。

本项目代码托管在Github上\footnote{项目主页\url{http://uestc-acm.github.io/CDOJ/}。},评测器内核位于branches文件夹,项目相关文档位于doc文件夹,服务器代码位于trunk文件夹下\footnote{项目最早托管于Google Code上,使用SVN作为版本控制工具,目录结构也是SVN风格的结构。}。评测器内核不在本文的讨论范围内,我们主要说明下trunk目录的结构。

trunk目录下的pom.xml是maven工程的配置文件,config和script用来保存一些编译脚本和配置文件,所有的代码均在src目录下。

\pic[htbp]{src目录结构}{width=\textwidth}{srcdirectorystructure}

java目录保存java源代码,resources文件夹保存系统配置信息,sql目录下保存数据库脚本,webapp下保存和前端相关的所有文件,包括样式表、图片、JSP文件。

\section{服务器端框架}
\subsection{服务器端总体技术架构}
为确保系统的互操作性、适用性及长期的扩充性,我们应以标准的、开放的要求进行架构。本文所设计的在线评测系统,是在优化、改造原有设计的基础上,借助于分布式的应用模型及先进的MVC体系结构实现的。在服务器端我们集中安置了系统中的数据库、程序和一些其它的组件,而在客户端我们只需要浏览器,同一数据源为用户提供数据查询服务,如此一来也就确保了数据的完整性与及时性。在很多情况下,用户的需要会随时间的推移而改变,因此在业务的处理逻辑出现变化的情况下,我们只需要在服务器端进行程序的修改,随后就可以重新进行发布,这样就方便了程序的研发及发布,也不会对用户产生影响。本系统总体结构如图\ref{Architecture}所示。
\pic[htbp]{系统总体技术架构}{}{Architecture}

客户端发起一个HTTP请求后,经过一系列中间处理最终被分配到该连接对应的控制器上(Controller)。与标准MVC模型不同的是,我们通过一个叫做服务(Service)的中间件来和模型(Model)进行交互,服务调用Hibernate框架的数据访问对象(DAO)来进行数据的持久化操作。在一系列逻辑操作之后,控制器(Controller)根据结果来选择合适的视图(View)返回给客户端。

评测器模块作为一个独立的模块存在于web框架之外,它通过服务(Service)来查找等待评测的任务队列、进行评测和更新任务队列。

\subsection{服务器端模块结构}\label{sec:serverModelStructure}
根据图\ref{Architecture}的结构图,服务器端的模块可以进一步细化为如下结构:
\begin{itemize}
	\item 配置模块(config):负责项目的整体配置。
	\item 数据库模块(db):使用Hibernate框架来完成持久化操作。
	\item 评测器模块(judge):负责评测任务的调度和执行。
	\item 服务模块(service):充当控制器(Controller)和模型(Model)的桥梁。
	\item 实用工具(util):提供许多有价值的API。
	\item 网站模块(web):包含了控制器(Controller)和许多与服务无关的组件。
\end{itemize}

\subsubsection{配置模块}
该模块承担的任务是对服务器的基本运行参数进行配置,例如Spring MVC框架的初始化配置、Hibernate框架的属性配置等。

\subsubsection{数据库模块}
该模块中包含了数据访问对象(DAO)以及相关的类,例如用于和数据库进行映射操作的实体类(Entity),可以转换为HQL语言\footnote{HQL是Hibernate Query Language的简写,即 hibernate 查询语言:HQL采用面向对象的查询方式。}查询条件的条件类(Condition)和数据传输对象(DTO)。

\subsubsection{评测器模块}
该模块包含了一个评测器服务(Judge Service),它产生许多个评测线程来进行多线程评测。

\subsubsection{服务模块}
这个模块的主要作用是提供丰富的模型操作API,例如模型实例的新建、修改、查找等操作。

\subsubsection{实用工具}
这个模块包含了所有供内部使用的公共API。

\subsubsection{网站模块}
该模块最主要的部分是控制器(Controller)模块,控制器负责处理来自客户端的HTTP请求,并通过一定的逻辑选择合适的服务(Service)来完成用户的请求,同时根据情况将合适的视图(View)返回给用户。

\section{浏览器端框架}
\subsection{浏览器端总体技术架构}
传统的Web应用允许用户端填写表单(form),当提交表单时就向Web服务器发送一个请求。服务器接收并处理传来的表单,然后送回一个新的网页,但这个做法浪费了许多带宽,因为在前后两个页面中的大部分HTML码往往是相同的。由于每次应用的沟通都需要向服务器发送请求,应用的回应时间依赖于服务器的回应时间。这导致了用户界面的回应比本机应用慢得多。

与此不同,AJAX(Asynchronous JavaScript and XML\footnote{实际上数据格式可以由JSON代替,进一步减少数据量,形成所谓的AJAJ。本系统使用的便是更加轻便的JSON数据。},异步的JavaScript与XML技术)应用可以仅向服务器发送并取回必须的数据,并在客户端采用JavaScript处理来自服务器的回应。因为在服务器和浏览器之间交换的数据大量减少(大约只有原来的5\%),服务器回应更快了。同时,很多的处理工作可以在发出请求的客户端机器上完成,因此Web服务器的负荷也减少了。

本系统的服务器端提供以下几种资源:
\begin{enumerate}
	\item 网页资源

	这类资源包含网页HTML代码、样式列表(CSS)、浏览器脚本(JavaScript)、图片和字体,通过GET方式获得。
	\item 数据资源

	这类资源均为JSON格式的数据,它有两种不同的获取方式:
	\begin{enumerate}
		\item GET方式

		这种方式和获取网页资源操作相同,区别在于服务器返回的是JSON数据。
		\item POST方式

		当客户端需要发送数据给服务器时,需要通过POST方式来传递数据,举个例子来说,当用户在登录窗口登录时,浏览器会将登录窗口的表单打包成一段JSON格式的数据,然后通过POST方式发送给服务器,服务器将登录状态等信息以JSON格式返回给前端,完成一次登录操作。
	\end{enumerate}
\end{enumerate}

在\ref{sec:angularjs}中我们提到了AngularJS框架,它是本系统最底层的框架。

\subsection{网站结构}
\pic[htbp]{网站结构图}{}{front-end-structure}

目前本项目域名为\url{http://acm.uestc.edu.cn/},根据功能需要我们将其划分若干部分,如图\ref{front-end-structure}所示。

\subsection{浏览器端开发流程}
\pic[htbp]{前端文件目录结构}{width=\textwidth}{cdojuidirectorystructure}

本系统所用到的样式表和脚本较多,这些文件统一使用GruntJS来维护,位于trunk/src/main/webapp/plugins/cdoj下。我们用LESS作为样式表的编程语言,CoffeeScript作为脚本的编程语言,编译流程如图\ref{GruntJS}所示。

\pic[htbp]{GruntJS编译活动图}{}{GruntJS}

项目中与AngularJS框架相关的脚本文件统一放置在src/angular子目录下。LESS文件在src/less下。还有部分功能我们用JQuery实现,位于src/jquery下。src/css和src/js被用来保存外部的CSS和JS文件。编译后的文件在dist目录下。

%\pic[htbp]{dist目录结构}{width=0.4\textwidth}{distdirectorystructure}