% !Mode:: "TeX:UTF-8"

\chapter{系统概要设计}
\section{系统环境}
在开始介绍系统的大体结构之前,首先介绍一下本系统的开发环境:
\begin{enumerate}
	\item 操作系统:OS X Mountain Lion 10.9.1
	\item 部署环境:运行于Virtual Box下的最新版本Arch Linux
	\item 建模软件:Visual Paradigm for UML
	\item 编辑软件:Eclipse、WebStorm、Vim
	\item 项目管理:Gradle\footnote{Gradle是一个基于Apache Ant和Apache Maven概念的项目自动化建构工具。它使用一种基于Groovy的特定领域语言来声明项目设置,而不是传统的XML。}
	\item 版本控制:Git
	\item 持续集成:TeamCity\footnote{一款功能强大的持续集成(Continue Integration)工具,可以让团队快速实现持续继承:IDE工具集成、各种消息通知、各种报表、项目的管理、分布式的编译等等。}
\end{enumerate}

本系统采用B/S架构,服务器部分采用的是JDK7.0标准下的Java语言,通过Gradle来组织和管理。Web端核心框架为AngularJS,UI采用Bootstrap3,项目通过Grunt来组织和管理,编程语言为Less\footnote{LESSCSS是一种动态样式语言,属于CSS预处理语言的一种,它使用类似CSS的语法,为CSS的赋予了动态语言的特性,如变量、继承、运算、函数等,更方便CSS的编写和维护。}和Coffeescript\footnote{CoffeeScript是一套JavaScript的转译语言。受到Ruby、Python与Haskell等语言的启发,CoffeeScript增强了JavaScript的简洁性与可读性。}。

本项目代码托管在Github上,项目主页:\url{http://uestc-acm.github.io/CDOJ/}。

\section{Web端技术架构}
传统的Web应用允许用户端填写表单(form),当提交表单时就向Web服务器发送一个请求。服务器接收并处理传来的表单,然后送回一个新的网页,但这个做法浪费了许多带宽,因为在前后两个页面中的大部分HTML码往往是相同的。由于每次应用的沟通都需要向服务器发送请求,应用的回应时间依赖于服务器的回应时间。这导致了用户界面的回应比本机应用慢得多。

与此不同,AJAX(Asynchronous JavaScript and XML\footnote{实际上数据格式可以由JSON代替,进一步减少数据量,形成所谓的AJAJ。本系统使用的便是更加轻便的JSON数据。},异步的JavaScript与XML技术)应用可以仅向服务器发送并取回必须的数据,并在客户端采用JavaScript处理来自服务器的回应。因为在服务器和浏览器之间交换的数据大量减少(大约只有原来的5\%),服务器回应更快了。同时,很多的处理工作可以在发出请求的客户端机器上完成,因此Web服务器的负荷也减少了。

Web端完全采用AJAX交互方式,和服务器通过JSON数据交互,具体方式有两种,GET和POST方式。GET方式直接向服务器发送请求,服务器返回JSON数据。当客户端需要发送数据给服务器时,需要通过POST方式来传递数据,举个例子来说,当用户在登录窗口登录时,浏览器会将登录窗口的表单打包成一段JSON格式的数据,然后通过POST方式发送给服务器,服务器将登录状态等信息以JSON格式返回给前端,完成一次登录操作。目前本项目域名为\url{http://acm.uestc.edu.cn/},Web端只有一个页面,既主页,页面切换功能由angularJS框架完成。

\pic[htbp]{前端文件目录结构}{width=\textwidth}{cdojuidirectorystructure}

\subsection{浏览器端开发流程}

本系统所用到的样式表和脚本较多,这些文件统一使用GruntJS来维护,位于trunk/src/main/webapp/cdoj下,编译流程如图\ref{GruntJS}所示。

\pic[htbp]{GruntJS编译活动图}{}{GruntJS}

项目中与AngularJS框架相关的脚本文件统一放置在src/angular子目录下,LESS文件在src/less下,相关依赖由bower管理,它们位于bower\_components文件夹,编译后的文件在dist目录下。

%Web端根据功能需要分成了五个部分,如图\ref{front-end-structure}所示。
%\subsection{网站结构}
%\pic[htbp]{网站结构图}{}{front-end-structure}
%
%根据功能需要将其划分若干部分,如图
%

\section{服务器技术架构}
为确保系统的互操作性、适用性及长期的扩充性,本系统应以标准的、开放的要求进行架构。本文所设计的在线评测系统,是在优化、改造原有设计的基础上,借助于分布式的应用模型及先进的MVC体系结构实现的。在服务器端集中安置了系统中的数据库、程序和一些其它的组件,而在客户端只需要浏览器,同一数据源为用户提供数据查询服务,如此一来也就确保了数据的完整性与及时性。在很多情况下,用户的需要会随时间的推移而改变,因此在业务的处理逻辑出现变化的情况下,只需要在服务器端进行程序的修改,随后就可以重新进行发布,这样就方便了程序的研发及发布,也不会对用户产生影响。本系统总体结构如图\ref{Architecture}所示。
\pic[htbp]{系统总体技术架构}{}{Architecture}

客户端发起一个HTTP请求后,经过一系列中间处理最终被分配到该连接对应的控制器上(Controller)。与标准MVC模型不同的是,本系统通过一个叫做服务(Service)的中间件来和模型(Model)进行交互,服务调用Hibernate框架的数据访问对象(DAO)来进行数据的持久化操作。在一系列逻辑操作之后,控制器(Controller)根据结果来选择合适的视图(View)返回给客户端。

评测器模块作为一个独立的模块存在于web框架之外,它通过服务(Service)来查找等待评测的任务队列、进行评测和更新任务队列。

\subsection{服务器端模块结构}\label{sec:serverModelStructure}
根据图\ref{Architecture}的结构图,服务器端的模块可以进一步细化为如下结构:
\begin{itemize}
	\item 配置模块(config):负责项目的整体配置。
	\item 数据库模块(db):使用Hibernate框架来完成持久化操作。
	\item 评测器模块(judge):负责评测任务的调度和执行。
	\item 服务模块(service):充当控制器(Controller)和模型(Model)的桥梁。
	\item 实用工具(util):提供许多有价值的API。
	\item 网站模块(web):包含了控制器(Controller)和许多与服务无关的组件。
\end{itemize}

\subsubsection{配置模块}
该模块承担的任务是对服务器的基本运行参数进行配置,例如Spring MVC框架的初始化配置、Hibernate框架的属性配置等。

\subsubsection{数据库模块}
该模块中包含了数据访问对象(DAO)以及相关的类,例如用于和数据库进行映射操作的实体类(Entity),可以转换为HQL语言\footnote{HQL是Hibernate Query Language的简写,即 hibernate 查询语言:HQL采用面向对象的查询方式。}查询条件的条件类(Condition)和数据传输对象(DTO)。

\subsubsection{评测器模块}
该模块包含了一个评测器服务(Judge Service),它产生许多个评测线程来进行多线程评测。

\subsubsection{服务模块}
这个模块的主要作用是提供丰富的模型操作API,例如模型实例的新建、修改、查找等操作。

\subsubsection{实用工具}
这个模块包含了所有供内部使用的公共API。

\subsubsection{网站模块}
该模块最主要的部分是控制器(Controller)模块,控制器负责处理来自客户端的HTTP请求,并通过一定的逻辑选择合适的服务(Service)来完成用户的请求,然后将结果返回给用户。
