% !Mode:: "TeX:UTF-8"

\chapter{系统详细设计}
\section{服务器端详细设计}
在服务器端系统设计过程中,最重要的是根据需求分析及用例模型构建系统静态模型和动态模型。顺序图展示对象之间的交互,这些交互是指在场景或用例的事件流中发生的。协作图是一种交互图,强调的是发送和接收消息的对象之间的组织结构,使用协作图来说明系统的动态情况。状态图说明对象在它的生命期中响应事件所经历的状态序列,以及它们对那些事件的响应。活动图是主要用于业务建模时,用于详述业务用例,描述一项业务的执行过程。设计时,描述操作的流程\cite{zhanghaipan1998}。

\subsection{系统顺序图}
我们用顺序图来说明用户的一次操作在系统之中是如何完成的,通过顺序图来对本系统的工作机制提供一个大概的说明,图\ref{BackendSequence}说明了用户的一次完整操作。

\pic[htbp]{系统顺序图}{width=\textwidth}{BackendSequence}

\subsection{系统包图}
包图说明了系统各个模块之间的依赖关系,在\ref{sec:serverModelStructure}中我们已经介绍过了系统的模块结构,根据这个结构,本系统的包结构如图\ref{ServerPackage}所示。由于本系统内容比较多,我们这里先给出大概的结构,后面再一一详细描述。
\pic[htb]{包图}{}{ServerPackage}

\subsection{Config Package详细设计}

\pic[htbp]{Config Package类图}{}{ConfigPackage}

\subsubsection{ApplicationContextConfig}
ApplicationContextConfig.java见附录\ref{sec:ApplicationContextConfig}。

Spring框架有两种配置方式,一种是通过XML配置文件进行配置,这种方式将所有的配置信息写入一个指定的XML文件之中,这种方式略显麻烦,在本文中我们采用了另外一种方式,这种方式是利用Java的Annotation机制来进行配置。

系统启动时默认将resources.properties文件\footnote{resources.properties见附录\ref{sec:resources}。}中的键值对初始化成一个Environment实例,我们可以通过getProperty(String): String方法来获得对应的值。

有了\textbf{Environment}实例,我们就可以将配置信息从代码中分离开来。

\subsubsection{WebMVCResource}
WebMVCResource.java见附录\ref{sec:WebMVCResource}。

为了方便进行测试,我们将一些比较特殊的资源从WebMVCConfig.java中独立开来,放到WebMVCResource.java中。这里主要做了两件事情:一个是配置视图解析器,在这里我们设置视图地址的前缀和后缀,方便Controller调用视图。另外一件事就是配置了JSON数据的转换器,用于解析和构建JSON数据,这里我们使用了fastjson\footnote{Fastjson是一个Java语言编写的JSON处理器,由阿里巴巴公司开发。}。

\subsubsection{WebMVCConfig}
WebMVCConfig.java见附录\ref{sec:WebMVCConfig}
这个文件是SpringMVC框架的配置文件,与之前的ApplicationContextConfig类似,这里配置了与Web相关的参数。

\subsection{Database Package详细设计}
\pic[htbp]{Database Package包图}{}{DatabasePackage}

这个包在MVC模型中处于Model层,所有与数据库有关的API都被包含在里面。

\subsubsection{Entity}
\pic[htbp]{Entity Package类图}{}{EntityPackage}

Entity即为实体,对应着MVC模型中的Model,它和数据库中的内容有着直接的一对一映射关系。本系统数据库较为复杂,详细数据库结构图见附录\ref{sec:databasediagram}。这里我们简述一下各个实体的作用,如表\ref{entitytable}所示

%\longthreelinetable{entitytable}{Entity表}{2}{ll}
\threelinetable[htbp]{entitytable}{\textwidth}{ll}{Entity表}
{Entity & 作用\\
}{
Article & 文章的内容和基本信息\\
Code & 用户提交的代码\\
CompileInfo & 代码的编译信息\\
Contest & 比赛的基本信息\\
ContestProblem & 比赛和题目的对应关系\\
ContestTeamInfo & 参赛队伍的信息\\
ContestUser & 比赛的注册用户\\
Department & 学校的部门信息\\
Language & 可以使用的语言以及参数\\
Message & 用户短消息\\
Problem & 题目内容和基本信息\\
ProblemTag & 题目和分类标签的对应关系\\
Status & 代码的评测状态\\
Tag & 分类标签\\
User & 用户信息\\
UserSerialKey & 用户激活码\\
}{
}

\subsubsection{DTO}
\pic[htbp]{DTO Package类图}{}{DTOPackage}

数据传输对象DTO有两种,一种是客户端向服务器传输的数据,一种是模型向上层传输的数据。前者我们通过一个简单的类可以实现。

Hibernate自带的数据库API较为复杂,为了提升效率和简化代码维护成本,我们自己构建了一套用来提取数据库数据的工具。在这类DTO中,我们使用了一个\textbf{@Fields}注解来注明这个DTO的信息来自数据库中的哪些域,然后通过这个field来构建HQL查询语言的\textbf{SELECT}命令。如下所示为UserListDTO.java的部分内容:

\noindent
\ttfamily
\hlstd{}\hllin{001\ }\hlkwc{@Fields}\hlstd{}\hlopt{(\{}\hlstd{}\hlstr{"userId"}\hlstd{}\hlopt{,\ }\hlstd{}\hlstr{"email"}\hlstd{}\hlopt{,\ }\hlstd{}\hlstr{"userName"}\hlstd{}\hlopt{,\ }\hlstd{}\hlstr{"nickName"}\hlstd{}\hlopt{,\ }\hlstd{}\hlstr{"type"}\hlstd{}\hlopt{,\ }\Righttorque\\
\hllin{002\ }\hlstd{}\hlstr{"school"}\hlstd{}\hlopt{,\ }\hlstd{}\hlstr{"motto"}\hlstd{}\hlopt{,\ }\hlstd{}\hlstr{"lastLogin"}\hlstd{}\hlopt{,\ }\hlstd{}\hlstr{"solved"}\hlstd{}\hlopt{,\ }\hlstd{}\hlstr{"tried"}\hlstd{}\hlopt{\})}\\
\hllin{003\ }\hlstd{}\hlkwa{public\ class\ }\hlstd{UserListDTO\ }\hlkwa{implements\ }\hlstd{BaseDTO}\hlopt{$<$}\hlstd{User}\hlopt{$>$\ \{}\\
\hllin{004\ }\hlstd{}\hlstd{\ \ }\hlstd{}\hlslc{//\ Codes}\\
\hllin{005\ }\hlstd{}\hlopt{\}}\hlstd{}\\
\mbox{}
\normalfont
\normalsize


对应生成的HQL语句为\textbf{SELECT userId, email, userName, nickName, type, school, motto, lastLogin, solved, tried FROM User},配合接下来要介绍到的Condition,我们可以组合出基本的HQL查询语句。

在得到这些域后,我们调用对应的EntityDTOBuilder的build方法来得到这些值。

\subsubsection{Condition}
\pic[htbp]{Condition Package类图}{}{ConditionPackage}

我们在本系统中使用Hibernate作为持久层框架,它提供了强大的HQL查询语言,Condition包的主要功能就是提供了Condition组件,它可以翻译成HQL查询语言的where条件,来限定检索范围。

根据实际情况,本系统设计的Condition支持三种条件:
\begin{enumerate}
	\item Order条件:用来限定返回结果的顺序。
	\item PageInfo条件:用来实现返回结果的分页功能。
	\item 普通条件:既Entry,它既可以是一条普通的条件,如\textbf{userId = 5},也可以是一个Condition。在枚举类型ConditionType中,我们定义了许多常用的条件,如等于、不等于、小于、like、属于等等。
\end{enumerate}

对于每个数据库实体类型Entity,都有一个对应的EntityCondition类,如Problem实体有对应的ProblemCondition。这些EntityCondition类都必须继承自BaseCondition类,并且实现它的\textbf{getCondition()}方法。

对于一些比较简单的条件,我们提供了一个\textbf{@Exp}注解,例如在StatusCondition.java中有如下变量:

\noindent
\ttfamily
\hlstd{}\hllin{001\ }\hlcom{/{*}{*}}\\
\hllin{002\ }\hlcom{\ {*}\ Minimal\ status\ id.}\\
\hllin{003\ }\hlcom{\ {*}/}\hlstd{}\\
\hllin{004\ }\hlkwc{@Exp}\hlstd{}\hlopt{(}\hlstd{mapField\ }\hlopt{=\ }\hlstd{}\hlstr{"statusId"}\hlstd{}\hlopt{,\ }\hlstd{type\ }\hlopt{=\ }\hlstd{Condition}\hlopt{.}\hlstd{ConditionType}\hlopt{.}\Righttorque\\
\hllin{005\ }\hlstd{GREATER\textunderscore OR\textunderscore EQUALS}\hlopt{)}\\
\hllin{006\ }\hlstd{}\hlkwa{public\ }\hlstd{Integer\ startId}\hlopt{;}\\
\hllin{007\ }\hlstd{}\\
\hllin{008\ }\hlcom{/{*}{*}}\\
\hllin{009\ }\hlcom{\ {*}\ Submit\ user\ id.}\\
\hllin{010\ }\hlcom{\ {*}/}\hlstd{}\\
\hllin{011\ }\hlkwc{@Exp}\hlstd{}\hlopt{(}\hlstd{mapField\ }\hlopt{=\ }\hlstd{}\hlstr{"userByUserId"}\hlstd{}\hlopt{,\ }\hlstd{type\ }\hlopt{=\ }\hlstd{Condition}\hlopt{.}\Righttorque\\
\hllin{012\ }\hlstd{ConditionType}\hlopt{.}\hlstd{EQUALS}\hlopt{)}\\
\hllin{013\ }\hlstd{}\hlkwa{public\ }\hlstd{Integer\ userId}\hlopt{;}\hlstd{}\\
\mbox{}
\normalfont
\normalsize


如果这两个成员变量不是空,那么最后我们会得到一个形式如同\textbf{WHERE ... userId $>=$ userId and userByUserId $=$ userId ...}的HQL查询语句。

对于一些比较复杂的条件,开发者可以在\textbf{getCondition()}方法中实现复杂的逻辑。

\subsubsection{DAO}
\pic[htbp]{DAO Package类图}{}{DAOPackage}

DAO提供了基础的数据库操作API,例如添加数据、修改、删除、查询等等,通过与DTO和Condition的配合使用,我们可以方便的进行数据库操作,而不需要为每种情况都生成一段冗长的HQL语句。

\subsection{Service Package详细设计}
\pic[htbp]{Service Package类图}{}{ServicePackage}

Service为上层应用提供了一系列特定的数据库操作,根据接口隔离原则\cite{szyperski2002component},我们不希望上层应用直接调用底层的数据库API来进行操作,我们通过Service来隔离它们。在这里,每个EntityService都完成与指定Entity相关的操作,不允许出现跨Entity的调用。

\subsection{Judge Package详细设计}
这个Package包含了与评测器服务相关的内容。

\subsubsection{评测器内核}
评测器内核负责编译、运行、评测用户代码,是一个控制台程序,通过命令行参数来设定评测任务。评测器内核的主函数参数表见表\ref{judgecoredescription}。
%\longthreelinetable{judgecoredescription}{Judge Core参数}{2}{ll}
\threelinetable[htbp]{judgecoredescription}{\textwidth}{ll}{Judge Core参数}
{参数 & 作用\\
}{
-u & 指定任务ID\\
-s & 指定源代码路径\\
-n & 指定题目ID\\
-D & 指定数据文件夹地址\\
-d & 指定运行的工作目录\\
-t & 指定运行时间限制\\
-m & 指定运行内存限制\\
-o & 指定输出大小限制\\
-S & 开启SPJ选项\\
-l & 指定语言类型\\
-I & 指定测试用例的输入文件\\
-O & 指定-I中测试用例的对应标准输出文件\\
-C & 是否需要编译\\
}{
}

评测结束后,它返回三个整数,分别代表评测结果、内存开销、时间开销。

\subsubsection{JudgeService}
\pic[htbp]{Judge Package类图}{}{JudgePackage}
\pic[htbp]{Judge Service活动图}{}{JudgeActivity}
JudgeService在系统启动时开始运行\footnote{见附录\ref{sec:ApplicationContextConfig}的22-26行。},在这个类中我们用队列judgeQueue作为评测器的调度队列。它生成schedulerThread线程用来等待评测任务的到来,它每隔一定的时间间隔(在这里我们设置为3秒)调用StatusService查找所有等待测试的任务,将其标记为OJ\_JUDGING状态,并加入到judgeQueue中。它还配置了若干个JudgeThread线程用来进行多线程评测操作。每个JudgeThread不停的扫描judgeQueue,直到任务的到来,它首先将代码保存至工作目录下,然后构造控制台命令调用\textbf{Runtime.getRuntime().exec(shellCommand)}来和评测器内核交互,并得到结果,然后依据结果来做出相应的更新。如图\ref{JudgeActivity}所示。

\subsection{Web Package详细设计}
\pic[htbp]{Web Package类图}{}{WebPackage}
Web Package主要包含的是控制器,以及为控制器服务的一些模块,比如权限验证模块。

\subsubsection{AuthenticationAspect}
这是本系统的权限验证模块,我们采用了面向侧面的程序设计\footnote{aspect-oriented programming,AOP,又译作面向方面的程序设计、观点导向编程,是计算机科学中的一个术语,指一种程序设计范型。该范型以一种称为侧面(aspect,又译作方面)的语言构造为基础,侧面是一种新的模块化机制,用来描述分散在对象、类或函数中的横切关注点(crosscutting concern)。}思想来完成,既在每个Controller之前``切入''一段指定的代码来进行权限验证。这部分我们使用AspectJ框架来完成,它是以代理(Proxy)的形式实现的。代码见附录\ref{sec:AuthenticationAspect}。

我们在图\ref{AuthenticationSequence}给出了验证成功和验证失败的两个顺序图,可以看到当验证失败时,我们不会调用原Controller,达到了权限限制的目的。

\begin{pics}[htbp]{权限验证顺序图}{AuthenticationSequence}
\addsubpic{一次失败的权限验证}{}{ControllerSequenceFail}
\addsubpic{一次成功的权限验证}{}{ControllerSequenceSuccess}
\end{pics}

\subsubsection{Controller}
Spring MVC框架提供了很强大的Controller,我们只需要在一个类上使用\textbf{@Controller}注解就可以将一个类声明为控制器。在本项目中,我们按照域名的分布来划分Controller,在前面的图\ref{front-end-structure}中我们已经给出网站的结构。

根据返回值的不同,控制器分为两种类型:一种是返回一个代表视图地址的\textbf{String}的控制器,一种是返回一个代表JSON数据的\textbf{@ResponseBody Map$<$String, Object$>$}的控制器。

视图保存在项目的webapp目录下,如\textbf{webapp/WEB-INF/views/index/index.jsp},此视图对应的地址为\textbf{/WEB-INF/views/index/index.jsp},但是我们在WebMVCResource.java(附录\ref{sec:WebMVCResource})中的\textbf{viewResolver()}设置了地址的前缀和后缀,我们可以将其简写为\textbf{index/index}。如果控制器返回了一个视图地址,那么服务器会将对应的页面返回给用户。

对于返回\textbf{@ResponseBody Map$<$String, Object$>$}的控制器,FastJson框架会将这个\textbf{@ResponseBody Map$<$String, Object$>$}转换成JSON各式的文本返回给用户。

下面我们用用户登陆相关的控制器来说明它是如何工作的(代码省略了其余部分)。

\noindent
\ttfamily
\hlstd{}\hllin{001\ }\hlslc{//\ 声明控制器}\\
\hllin{002\ }\hlstd{}\hlkwc{@Controller}\\
\hllin{003\ }\hlstd{}\hlslc{//\ 声明控制器的域名}\\
\hllin{004\ }\hlstd{}\hlkwc{@RequestMapping}\hlstd{}\hlopt{(}\hlstd{}\hlstr{"/user"}\hlstd{}\hlopt{)}\\
\hllin{005\ }\hlstd{}\hlkwa{public\ class\ }\hlstd{UserController\ }\hlkwa{extends\ }\hlstd{BaseController\ }\hlopt{\{}\\
\hllin{006\ }\hlstd{\\
\hllin{007\ }}\hlstd{\ \ }\hlstd{}\hlslc{//\ /user/login对应的方法}\\
\hllin{008\ }\hlstd{}\hlstd{\ \ }\hlstd{}\hlkwc{@RequestMapping}\hlstd{}\hlopt{(}\hlstd{}\hlstr{"login"}\hlstd{}\hlopt{)}\\
\hllin{009\ }\hlstd{}\hlstd{\ \ }\hlstd{}\hlslc{//\ 登陆权限设置为无}\\
\hllin{010\ }\hlstd{}\hlstd{\ \ }\hlstd{}\hlkwc{@LoginPermit}\hlstd{}\hlopt{(}\hlstd{NeedLogin\ }\hlopt{=\ }\hlstd{false}\hlopt{)}\\
\hllin{011\ }\hlstd{}\hlstd{\ \ }\hlstd{}\hlkwa{public}\\
\hllin{012\ }\hlstd{}\hlstd{\ \ }\hlstd{}\hlslc{//\ 返回值为JSON数据}\\
\hllin{013\ }\hlstd{}\hlstd{\ \ }\hlstd{}\hlkwc{@ResponseBody}\\
\hllin{014\ }\hlstd{}\hlstd{\ \ }\hlstd{Map}\hlopt{$<$}\hlstd{String}\hlopt{,\ }\hlstd{Object}\hlopt{$>$\ }\hlstd{}\hlkwd{login}\hlstd{}\hlopt{(}\hlstd{HttpSession\ session}\hlopt{,}\\
\hllin{015\ }\hlstd{}\hlstd{\ \ \ \ \ \ \ \ \ \ \ \ \ \ \ \ \ \ \ \ \ \ \ \ \ \ \ \ }\hlstd{}\hlkwc{@RequestBody\ @Valid\ }\Righttorque\\
\hllin{016\ }\hlstd{}\hlstd{\ \ \ \ \ \ \ \ \ \ \ \ \ \ \ \ \ \ \ \ \ \ \ \ \ \ \ \ }\hlstd{UserLoginDTO\ userLoginDTO}\hlopt{,}\\
\hllin{017\ }\hlstd{}\hlstd{\ \ \ \ \ \ \ \ \ \ \ \ \ \ \ \ \ \ \ \ \ \ \ \ \ \ \ \ }\hlstd{BindingResult\ validateResult}\hlopt{)\ \{}\\
\hllin{018\ }\hlstd{}\hlstd{\ \ \ \ }\hlstd{Map}\hlopt{$<$}\hlstd{String}\hlopt{,\ }\hlstd{Object}\hlopt{$>$\ }\hlstd{json\ }\hlopt{=\ }\hlstd{}\hlkwa{new\ }\hlstd{HashMap}\hlopt{$<$$>$();}\\
\hllin{019\ }\hlstd{}\hlstd{\ \ \ \ }\hlstd{}\hlslc{//\ 表单验证失败}\\
\hllin{020\ }\hlstd{}\hlstd{\ \ \ \ }\hlstd{}\hlkwa{if\ }\hlstd{}\hlopt{(}\hlstd{validateResult}\hlopt{.}\hlstd{}\hlkwd{hasErrors}\hlstd{}\hlopt{())\ \{}\\
\hllin{021\ }\hlstd{}\hlstd{\ \ \ \ \ \ }\hlstd{json}\hlopt{.}\hlstd{}\hlkwd{put}\hlstd{}\hlopt{(}\hlstd{}\hlstr{"result"}\hlstd{}\hlopt{,\ }\hlstd{}\hlstr{"field\textunderscore error"}\hlstd{}\hlopt{);}\\
\hllin{022\ }\hlstd{}\hlstd{\ \ \ \ \ \ }\hlstd{json}\hlopt{.}\hlstd{}\hlkwd{put}\hlstd{}\hlopt{(}\hlstd{}\hlstr{"field"}\hlstd{}\hlopt{,\ }\hlstd{validateResult}\hlopt{.}\hlstd{}\hlkwd{getFieldErrors}\hlstd{}\hlopt{());}\\
\hllin{023\ }\hlstd{}\hlstd{\ \ \ \ }\hlstd{}\hlopt{\}\ }\hlstd{}\hlkwa{else\ }\hlstd{}\hlopt{\{}\\
\hllin{024\ }\hlstd{}\hlstd{\ \ \ \ \ \ }\hlstd{}\hlkwa{try\ }\hlstd{}\hlopt{\{}\\
\hllin{025\ }\hlstd{}\hlstd{\ \ \ \ \ \ \ \ }\hlstd{}\hlslc{//\ 获取登陆用户的信息}\\
\hllin{026\ }\hlstd{}\hlstd{\ \ \ \ \ \ \ \ }\hlstd{UserDTO\ userDTO\ }\hlopt{=\ }\hlstd{userService}\hlopt{.}\hlstd{}\hlkwd{getUserDTOByUserName}\hlstd{}\hlopt{(}\Righttorque\\
\hllin{027\ }\hlstd{}\hlstd{\ \ \ \ \ \ \ \ }\hlstd{userLoginDTO}\hlopt{.}\hlstd{}\hlkwd{getUserName}\hlstd{}\hlopt{());}\\
\hllin{028\ }\hlstd{}\hlstd{\ \ \ \ \ \ \ \ }\hlstd{}\hlslc{//\ 密码验证}\\
\hllin{029\ }\hlstd{}\hlstd{\ \ \ \ \ \ \ \ }\hlstd{}\hlkwa{if\ }\hlstd{}\hlopt{(}\hlstd{userDTO\ }\hlopt{==\ }\hlstd{null\ }\hlopt{\textbar \textbar \ !}\hlstd{userLoginDTO}\hlopt{.}\hlstd{}\hlkwd{getPassword}\hlstd{}\hlopt{().}\Righttorque\\
\hllin{030\ }\hlstd{}\hlstd{\ \ \ \ \ \ \ \ }\hlstd{}\hlkwd{equals}\hlstd{}\hlopt{(}\hlstd{userDTO}\hlopt{.}\hlstd{}\hlkwd{getPassword}\hlstd{}\hlopt{()))\ \{}\\
\hllin{031\ }\hlstd{}\hlstd{\ \ \ \ \ \ \ \ \ \ }\hlstd{}\hlkwa{throw\ new\ }\hlstd{}\hlkwd{FieldException}\hlstd{}\hlopt{(}\hlstd{}\hlstr{"password"}\hlstd{}\hlopt{,\ }\hlstd{}\hlstr{"User\ or\ }\Righttorque\\
\hllin{032\ }\hlstr{}\hlstd{\ \ \ \ \ \ \ \ \ \ }\hlstr{password\ is\ wrong,\ please\ try\ again"}\hlstd{}\hlopt{);}\\
\hllin{033\ }\hlstd{}\hlstd{\ \ \ \ \ \ \ \ }\hlstd{}\hlopt{\}}\\
\hllin{034\ }\hlstd{}\hlstd{\ \ \ \ \ \ \ \ }\hlstd{}\hlslc{//\ 更新用户}\\
\hllin{035\ }\hlstd{}\hlstd{\ \ \ \ \ \ \ \ }\hlstd{userDTO}\hlopt{.}\hlstd{}\hlkwd{setLastLogin}\hlstd{}\hlopt{(}\hlstd{}\hlkwa{new\ }\hlstd{}\hlkwd{Timestamp}\hlstd{}\hlopt{(}\hlstd{}\hlkwa{new\ }\hlstd{}\hlkwd{Date}\hlstd{}\hlopt{().}\Righttorque\\
\hllin{036\ }\hlstd{}\hlstd{\ \ \ \ \ \ \ \ }\hlstd{}\hlkwd{getTime}\hlstd{}\hlopt{()\ /\ }\hlstd{}\hlnum{1000\ }\hlstd{}\hlopt{{*}\ }\hlstd{}\hlnum{1000}\hlstd{}\hlopt{));}\\
\hllin{037\ }\hlstd{}\hlstd{\ \ \ \ \ \ \ \ }\hlstd{userService}\hlopt{.}\hlstd{}\hlkwd{updateUser}\hlstd{}\hlopt{(}\hlstd{userDTO}\hlopt{);}\\
\hllin{038\ }\hlstd{\\
\hllin{039\ }}\hlstd{\ \ \ \ \ \ \ \ }\hlstd{}\hlslc{//\ 将用户信息放入Session中}\\
\hllin{040\ }\hlstd{}\hlstd{\ \ \ \ \ \ \ \ }\hlstd{session}\hlopt{.}\hlstd{}\hlkwd{setAttribute}\hlstd{}\hlopt{(}\hlstd{}\hlstr{"currentUser"}\hlstd{}\hlopt{,\ }\hlstd{userDTO}\hlopt{);}\\
\hllin{041\ }\hlstd{\\
\hllin{042\ }}\hlstd{\ \ \ \ \ \ \ \ }\hlstd{}\hlslc{//\ 构造返回数据}\\
\hllin{043\ }\hlstd{}\hlstd{\ \ \ \ \ \ \ \ }\hlstd{json}\hlopt{.}\hlstd{}\hlkwd{put}\hlstd{}\hlopt{(}\hlstd{}\hlstr{"userName"}\hlstd{}\hlopt{,\ }\hlstd{userDTO}\hlopt{.}\hlstd{}\hlkwd{getUserName}\hlstd{}\hlopt{());}\\
\hllin{044\ }\hlstd{}\hlstd{\ \ \ \ \ \ \ \ }\hlstd{json}\hlopt{.}\hlstd{}\hlkwd{put}\hlstd{}\hlopt{(}\hlstd{}\hlstr{"type"}\hlstd{}\hlopt{,\ }\hlstd{userDTO}\hlopt{.}\hlstd{}\hlkwd{getType}\hlstd{}\hlopt{());}\\
\hllin{045\ }\hlstd{}\hlstd{\ \ \ \ \ \ \ \ }\hlstd{json}\hlopt{.}\hlstd{}\hlkwd{put}\hlstd{}\hlopt{(}\hlstd{}\hlstr{"email"}\hlstd{}\hlopt{,\ }\hlstd{userDTO}\hlopt{.}\hlstd{}\hlkwd{getEmail}\hlstd{}\hlopt{());}\\
\hllin{046\ }\hlstd{}\hlstd{\ \ \ \ \ \ \ \ }\hlstd{json}\hlopt{.}\hlstd{}\hlkwd{put}\hlstd{}\hlopt{(}\hlstd{}\hlstr{"result"}\hlstd{}\hlopt{,\ }\hlstd{}\hlstr{"success"}\hlstd{}\hlopt{);}\\
\hllin{047\ }\hlstd{}\hlstd{\ \ \ \ \ \ }\hlstd{}\hlopt{\}\ }\hlstd{}\hlkwa{catch\ }\hlstd{}\hlopt{(}\hlstd{FieldException\ e}\hlopt{)\ \{}\\
\hllin{048\ }\hlstd{}\hlstd{\ \ \ \ \ \ \ \ }\hlstd{}\hlslc{//\ 如果存在表单错误}\\
\hllin{049\ }\hlstd{}\hlstd{\ \ \ \ \ \ \ \ }\hlstd{}\hlkwd{putFieldErrorsIntoBindingResult}\hlstd{}\hlopt{(}\hlstd{e}\hlopt{,\ }\hlstd{validateResult}\hlopt{);}\\
\hllin{050\ }\hlstd{}\hlstd{\ \ \ \ \ \ \ \ }\hlstd{json}\hlopt{.}\hlstd{}\hlkwd{put}\hlstd{}\hlopt{(}\hlstd{}\hlstr{"result"}\hlstd{}\hlopt{,\ }\hlstd{}\hlstr{"field\textunderscore error"}\hlstd{}\hlopt{);}\\
\hllin{051\ }\hlstd{}\hlstd{\ \ \ \ \ \ \ \ }\hlstd{json}\hlopt{.}\hlstd{}\hlkwd{put}\hlstd{}\hlopt{(}\hlstd{}\hlstr{"field"}\hlstd{}\hlopt{,\ }\hlstd{validateResult}\hlopt{.}\hlstd{}\hlkwd{getFieldErrors}\hlstd{}\hlopt{());}\\
\hllin{052\ }\hlstd{}\hlstd{\ \ \ \ \ \ }\hlstd{}\hlopt{\}\ }\hlstd{}\hlkwa{catch\ }\hlstd{}\hlopt{(}\hlstd{AppException\ e}\hlopt{)\ \{}\\
\hllin{053\ }\hlstd{}\hlstd{\ \ \ \ \ \ \ \ }\hlstd{}\hlslc{//\ 如果发生其它错误}\\
\hllin{054\ }\hlstd{}\hlstd{\ \ \ \ \ \ \ \ }\hlstd{json}\hlopt{.}\hlstd{}\hlkwd{put}\hlstd{}\hlopt{(}\hlstd{}\hlstr{"result"}\hlstd{}\hlopt{,\ }\hlstd{}\hlstr{"error"}\hlstd{}\hlopt{);}\\
\hllin{055\ }\hlstd{}\hlstd{\ \ \ \ \ \ \ \ }\hlstd{json}\hlopt{.}\hlstd{}\hlkwd{put}\hlstd{}\hlopt{(}\hlstd{}\hlstr{"error\textunderscore msg"}\hlstd{}\hlopt{,\ }\hlstd{e}\hlopt{.}\hlstd{}\hlkwd{getMessage}\hlstd{}\hlopt{());}\\
\hllin{056\ }\hlstd{}\hlstd{\ \ \ \ \ \ }\hlstd{}\hlopt{\}}\\
\hllin{057\ }\hlstd{}\hlstd{\ \ \ \ }\hlstd{}\hlopt{\}}\\
\hllin{058\ }\hlstd{}\hlstd{\ \ \ \ }\hlstd{}\hlslc{//\ 返回结果}\\
\hllin{059\ }\hlstd{}\hlstd{\ \ \ \ }\hlstd{}\hlkwa{return\ }\hlstd{json}\hlopt{;}\\
\hllin{060\ }\hlstd{}\hlstd{\ \ }\hlstd{}\hlopt{\}}\\
\hllin{061\ }\hlstd{}\\
\hllin{062\ }\hlopt{\}}\hlstd{}\\
\mbox{}
\normalfont
\normalsize


我们使用\textbf{@RequestMap}注解来声明方法所对应的网址,\textbf{@LoginPermit}定义了该地址的权限。因为Spring通过反射机制来进行自动注入,控制器方法的参数可以以任意顺序排列而不需要指定顺序。在login方法中,除了\textbf{HttpSession session}参数以外,另外两个和前端POST来的数据相关,\textbf{@RequestBody @Valid UserLoginDTO userLoginDTO}用来保存前端POST来的数据,我们使用了Java验证框架来对前端传递的数据进行一个初步的合法性验证,\textbf{@Valid}注解说明了我们需要对\textbf{userLoginDTO}进行验证,验证的结果保存在\textbf{BindingResult validateResult}中,下面是\textbf{UserLoginDTO}的部分代码:

\noindent
\ttfamily
\hlstd{}\hllin{001\ }\hlcom{/{*}{*}}\\
\hllin{002\ }\hlcom{\ {*}\ DTO\ post\ from\ user\ login\ form.}\\
\hllin{003\ }\hlcom{\ {*}/}\hlstd{}\\
\hllin{004\ }\hlkwa{public\ class\ }\hlstd{UserLoginDTO\ }\hlopt{\{}\\
\hllin{005\ }\hlstd{\\
\hllin{006\ }}\hlstd{\ \ }\hlstd{}\hlcom{/{*}{*}}\\
\hllin{007\ }\hlcom{}\hlstd{\ \ \ }\hlcom{{*}\ Input:\ user\ name}\\
\hllin{008\ }\hlcom{}\hlstd{\ \ \ }\hlcom{{*}/}\hlstd{\\
\hllin{009\ }}\hlstd{\ \ }\hlstd{}\hlslc{//\ 非空验证}\\
\hllin{010\ }\hlstd{}\hlstd{\ \ }\hlstd{}\hlkwc{@NotNull}\hlstd{}\hlopt{(}\hlstd{message\ }\hlopt{=\ }\hlstd{}\hlstr{"Please\ enter\ your\ user\ name."}\hlstd{}\hlopt{)}\\
\hllin{011\ }\hlstd{}\hlstd{\ \ }\hlstd{}\hlslc{//\ 正则表达式验证}\\
\hllin{012\ }\hlstd{}\hlstd{\ \ }\hlstd{}\hlkwc{@Pattern}\hlstd{}\hlopt{(}\hlstd{regexp\ }\hlopt{=\ }\hlstd{}\hlstr{"}\hlesc{$\backslash$$\backslash$}\hlstr{b\textasciicircum {[}a{-}zA{-}Z0{-}9\textunderscore {]}\{4,24\}\$}\hlesc{$\backslash$$\backslash$}\hlstr{b"}\hlstd{}\hlopt{,}\\
\hllin{013\ }\hlstd{}\hlstd{\ \ \ \ \ \ }\hlstd{message\ }\hlopt{=\ }\hlstd{}\hlstr{"Please\ enter\ 4{-}24\ characters\ consist\ of\ A{-}}\Righttorque\\
\hllin{014\ }\hlstr{}\hlstd{\ \ \ \ \ \ }\hlstr{Z,\ a{-}z,\ 0{-}9\ and\ '\textunderscore '."}\hlstd{}\hlopt{)}\\
\hllin{015\ }\hlstd{}\hlstd{\ \ }\hlstd{}\hlkwa{private\ }\hlstd{String\ userName}\hlopt{;}\\
\hllin{016\ }\hlstd{\\
\hllin{017\ }}\hlstd{\ \ }\hlstd{}\hlcom{/{*}{*}}\\
\hllin{018\ }\hlcom{}\hlstd{\ \ \ }\hlcom{{*}\ Input:\ password}\\
\hllin{019\ }\hlcom{}\hlstd{\ \ \ }\hlcom{{*}/}\hlstd{\\
\hllin{020\ }}\hlstd{\ \ }\hlstd{}\hlslc{//\ 非空验证}\\
\hllin{021\ }\hlstd{}\hlstd{\ \ }\hlstd{}\hlkwc{@NotNull}\hlstd{}\hlopt{(}\hlstd{message\ }\hlopt{=\ }\hlstd{}\hlstr{"Please\ enter\ your\ password."}\hlstd{}\hlopt{)}\\
\hllin{022\ }\hlstd{}\hlstd{\ \ }\hlstd{}\hlslc{//\ 长度验证}\\
\hllin{023\ }\hlstd{}\hlstd{\ \ }\hlstd{}\hlkwc{@Length}\hlstd{}\hlopt{(}\hlstd{min\ }\hlopt{=\ }\hlstd{}\hlnum{40}\hlstd{}\hlopt{,\ }\hlstd{max\ }\hlopt{=\ }\hlstd{}\hlnum{40}\hlstd{}\hlopt{,\ }\hlstd{message\ }\hlopt{=\ }\hlstd{}\hlstr{"Please\ enter\ your\ }\Righttorque\\
\hllin{024\ }\hlstr{}\hlstd{\ \ }\hlstr{password."}\hlstd{}\hlopt{)}\\
\hllin{025\ }\hlstd{}\hlstd{\ \ }\hlstd{}\hlkwa{private\ }\hlstd{String\ password}\hlopt{;}\\
\hllin{026\ }\hlstd{}\\
\hllin{027\ }\hlopt{\}}\hlstd{}\\
\mbox{}
\normalfont
\normalsize


顺利登陆之后,前端可以接收到类似如下格式的一段数据:

\noindent
\ttfamily
\hlstd{\hllin{001\ }\{"email":"muziriyun@qq.com","result":"success","type":1,\Righttorque\\
\hllin{002\ }"userName":"UESTC\textunderscore Izayoi"\}}\\
\mbox{}
\normalfont
\normalsize


\subsubsection{网站地图}
上面我们举例说明了UserController的login方法的实现,限于篇幅限制其余的部分我们不一一描述,在这里我们给出整个网站的网站地图,介绍各个控制器的作用和返回给前端的数据类型。

\longthreelinetable{sitemap}{网站地图}{5}{lllll}
{
地址  & 控制器 & 方法 & 作用 & 返回类型\\
}{
/admin/ & AdminController & index & 管理员面板 & HTML\\

/article/data/{type}/{articleId} & ArticleController & data & 文章数据 & JSON\\
/article/show/{articleId} & ArticleController & show & 文章页面 & HTML\\
/article/search & ArticleController & search & 文章查找 & JSON\\
/article/editor/{articleId} & ArticleController & editor & 文章编辑器 & HTML\\
/article/edit & ArticleController & edit & 文章编辑 & JSON\\
/article/operator & ArticleController & operator & 文章操作 & JSON\\

/contest/status/{contestId}/{lastFetched} & ContestController & status & 比赛评测结果 & JSON\\
/contest/rankList/{contestId} & ContestController & rankList & 比赛排名 & JSON\\
/contest/data/{contestId} & ContestController & data & 比赛数据 & JSON\\
/contest/show/{contestId} & ContestController & show & 比赛页面 & HTML\\
/contest/list/{contestId} & ContestController & list & 比赛列表 & HTML\\
/contest/search & ContestController & search & 比赛查找 & JSON\\
/contest/operator/{id}/{field}/{value} & ContestController & operator & 比赛操作 & JSON\\
/contest/editor/{contestId} & ContestController & editor & 比赛编辑器 & HTML\\
/contest/edit & ContestController & edit & 比赛编辑 & JSON\\
}

\section{浏览器端详细设计}
一个好的Web应用不仅仅要拥有功能完善的后台,还应该拥有一个友好的界面。下面我们