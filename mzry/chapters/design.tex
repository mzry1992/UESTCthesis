% !Mode:: "TeX:UTF-8"

\chapter{系统详细设计}
\section{服务器端详细设计}
在系统设计过程中,最重要的是根据需求分析及用例模型构建系统静态模型和动态模型。顺序图展示对象之间的交互,这些交互是指在场景或用例的事件流中发生的。协作图是一种交互图,强调的是发送和接收消息的对象之间的组织结构,使用协作图来说明系统的动态情况。状态图说明对象在它的生命期中响应事件所经历的状态序列,以及它们对那些事件的响应。活动图是主要用于业务建模时,用于详述业务用例,描述一项业务的执行过程。设计时,描述操作的流程\cite{zhanghaipan1998}。

\subsection{系统包图}
包图说明了系统各个模块之间的依赖关系,在\ref{sec:serverModelStructure}中我们已经介绍过了系统的模块结构,根据这个结构,本系统的包结构如图\ref{ServerPackage}所示。由于本系统内容比较多,我们这里先给出大概的结构,后面再一一详细描述。
\pic[htbp]{包图}{}{ServerPackage}

\subsection{Config Package详细设计}

\pic[htbp]{Config Package类图}{}{ConfigPackage}

\subsubsection{ApplicationContextConfig}
Spring框架有两种配置方式,一种是通过XML配置文件进行配置,这种方式将所有的配置信息写入一个指定的XML文件之中,这种方式略显麻烦,在本文中我们采用了另外一种方式,这种方式是利用Java的Annotation机制来进行配置。

下面是ApplicationContextConfig.java的主要内容:

\noindent
\ttfamily
\hlstd{}\hllin{001\ }\hlcom{/{*}{*}}\\
\hllin{002\ }\hlcom{\ {*}\ Application\ Context\ Configuration.}\\
\hllin{003\ }\hlcom{\ {*}/}\hlstd{}\\
\hllin{004\ }\hlkwc{@Configuration}\\
\hllin{005\ }\hlstd{}\hlslc{//\ 设置Spring需要扫描的目录}\\
\hllin{006\ }\hlstd{}\hlkwc{@ComponentScan}\hlstd{}\hlopt{(}\hlstd{basePackages\ }\hlopt{=\ \{}\\
\hllin{007\ }\hlstd{}\hlstd{\ \ \ \ }\hlstd{}\hlstr{"cn.edu.uestc.acmicpc.db"}\hlstd{}\hlopt{,}\\
\hllin{008\ }\hlstd{}\hlstd{\ \ \ \ }\hlstd{}\hlstr{"cn.edu.uestc.acmicpc.judge"}\hlstd{}\hlopt{,}\\
\hllin{009\ }\hlstd{}\hlstd{\ \ \ \ }\hlstd{}\hlstr{"cn.edu.uestc.acmicpc.util"}\hlstd{}\hlopt{,}\\
\hllin{010\ }\hlstd{}\hlstd{\ \ \ \ }\hlstd{}\hlstr{"cn.edu.uestc.acmicpc.service"}\hlstd{}\hlopt{,}\\
\hllin{011\ }\hlstd{}\hlstd{\ \ \ \ }\hlstd{}\hlstr{"cn.edu.uestc.acmicpc.web.aspect"}\hlstd{}\\
\hllin{012\ }\hlopt{\})}\\
\hllin{013\ }\hlstd{}\hlslc{//\ 设置resources文件的地址}\\
\hllin{014\ }\hlstd{}\hlkwc{@PropertySource}\hlstd{}\hlopt{(}\hlstd{}\hlstr{"classpath:resources.properties"}\hlstd{}\hlopt{)}\\
\hllin{015\ }\hlstd{}\hlslc{//\ 开启事务管理}\\
\hllin{016\ }\hlstd{}\hlkwc{@EnableTransactionManagement}\\
\hllin{017\ }\hlstd{}\hlkwa{public\ class\ }\hlstd{ApplicationContextConfig\ }\hlopt{\{}\\
\hllin{018\ }\hlstd{\\
\hllin{019\ }}\hlstd{\ \ }\hlstd{}\hlkwc{@Autowired}\\
\hllin{020\ }\hlstd{}\hlstd{\ \ }\hlstd{}\hlkwa{private\ }\hlstd{Environment\ environment}\hlopt{;}\\
\hllin{021\ }\hlstd{\\
\hllin{022\ }}\hlstd{\ \ }\hlstd{}\hlcom{/{*}{*}}\\
\hllin{023\ }\hlcom{}\hlstd{\ \ \ }\hlcom{{*}\ Bean:\ Judge\ service.}\\
\hllin{024\ }\hlcom{}\hlstd{\ \ \ }\hlcom{{*}}\\
\hllin{025\ }\hlcom{}\hlstd{\ \ \ }\hlcom{{*}\ JudgeService\ is\ a\ singleton\ instance\ and\ need\ start\ at\ }\Righttorque\\
\hllin{026\ }\hlcom{}\hlstd{\ \ \ }\hlcom{first.}\\
\hllin{027\ }\hlcom{}\hlstd{\ \ \ }\hlcom{{*}}\\
\hllin{028\ }\hlcom{}\hlstd{\ \ \ }\hlcom{{*}\ @return\ judgeService\ bean}\\
\hllin{029\ }\hlcom{}\hlstd{\ \ \ }\hlcom{{*}/}\hlstd{\\
\hllin{030\ }}\hlstd{\ \ }\hlstd{}\hlslc{//\ 对Judge\ Service的配置}\\
\hllin{031\ }\hlstd{}\hlstd{\ \ }\hlstd{}\hlkwc{@Bean}\hlstd{}\hlopt{(}\hlstd{name\ }\hlopt{=\ }\hlstd{}\hlstr{"judgeService"}\hlstd{}\hlopt{)}\\
\hllin{032\ }\hlstd{}\hlstd{\ \ }\hlstd{}\hlkwc{@Scope}\hlstd{}\hlopt{(}\hlstd{ConfigurableBeanFactory}\hlopt{.}\hlstd{SCOPE\textunderscore SINGLETON}\hlopt{)}\\
\hllin{033\ }\hlstd{}\hlstd{\ \ }\hlstd{}\hlkwc{@Lazy}\hlstd{}\hlopt{(}\hlstd{false}\hlopt{)}\\
\hllin{034\ }\hlstd{}\hlstd{\ \ }\hlstd{}\hlkwa{public\ }\hlstd{JudgeService\ }\hlkwd{judgeService}\hlstd{}\hlopt{()\ \{}\\
\hllin{035\ }\hlstd{}\hlstd{\ \ \ \ }\hlstd{}\hlkwa{return\ new\ }\hlstd{}\hlkwd{JudgeService}\hlstd{}\hlopt{();}\\
\hllin{036\ }\hlstd{}\hlstd{\ \ }\hlstd{}\hlopt{\}}\\
\hllin{037\ }\hlstd{\\
\hllin{038\ }}\hlstd{\ \ }\hlstd{}\hlcom{/{*}{*}}\\
\hllin{039\ }\hlcom{}\hlstd{\ \ \ }\hlcom{{*}\ Bean:\ Data\ source}\\
\hllin{040\ }\hlcom{}\hlstd{\ \ \ }\hlcom{{*}}\\
\hllin{041\ }\hlcom{}\hlstd{\ \ \ }\hlcom{{*}\ We\ use\ BoneCP\ to\ manage\ connection\ poll.}\\
\hllin{042\ }\hlcom{}\hlstd{\ \ \ }\hlcom{{*}}\\
\hllin{043\ }\hlcom{}\hlstd{\ \ \ }\hlcom{{*}\ @return\ dataSource\ bean}\\
\hllin{044\ }\hlcom{}\hlstd{\ \ \ }\hlcom{{*}/}\hlstd{\\
\hllin{045\ }}\hlstd{\ \ }\hlstd{}\hlslc{//\ 对数据源的配置}\\
\hllin{046\ }\hlstd{}\hlstd{\ \ }\hlstd{}\hlkwc{@Bean}\hlstd{}\hlopt{(}\hlstd{name\ }\hlopt{=\ }\hlstd{}\hlstr{"dataSource"}\hlstd{}\hlopt{,\ }\hlstd{destroyMethod\ }\hlopt{=\ }\hlstd{}\hlstr{"close"}\hlstd{}\hlopt{)}\\
\hllin{047\ }\hlstd{}\hlstd{\ \ }\hlstd{}\hlkwa{public\ }\hlstd{BoneCPDataSource\ }\hlkwd{dataSource}\hlstd{}\hlopt{()\ \{}\\
\hllin{048\ }\hlstd{}\hlstd{\ \ \ \ }\hlstd{BoneCPDataSource\ dataSource\ }\hlopt{=\ }\hlstd{}\hlkwa{new\ }\hlstd{}\hlkwd{BoneCPDataSource}\hlstd{}\hlopt{();}\\
\hllin{049\ }\hlstd{\\
\hllin{050\ }}\hlstd{\ \ \ \ }\hlstd{dataSource}\hlopt{.}\hlstd{}\hlkwd{setDriverClass}\hlstd{}\hlopt{(}\hlstd{}\hlkwd{getProperty}\hlstd{}\hlopt{(}\hlstd{}\hlstr{"db.driver"}\hlstd{}\hlopt{));}\\
\hllin{051\ }\hlstd{}\hlstd{\ \ \ \ }\hlstd{dataSource}\hlopt{.}\hlstd{}\hlkwd{setJdbcUrl}\hlstd{}\hlopt{(}\hlstd{}\hlkwd{getProperty}\hlstd{}\hlopt{(}\hlstd{}\hlstr{"db.url"}\hlstd{}\hlopt{));}\\
\hllin{052\ }\hlstd{}\hlstd{\ \ \ \ }\hlstd{dataSource}\hlopt{.}\hlstd{}\hlkwd{setUsername}\hlstd{}\hlopt{(}\hlstd{}\hlkwd{getProperty}\hlstd{}\hlopt{(}\hlstd{}\hlstr{"db.username"}\hlstd{}\hlopt{));}\\
\hllin{053\ }\hlstd{}\hlstd{\ \ \ \ }\hlstd{dataSource}\hlopt{.}\hlstd{}\hlkwd{setPassword}\hlstd{}\hlopt{(}\hlstd{}\hlkwd{getProperty}\hlstd{}\hlopt{(}\hlstd{}\hlstr{"db.password"}\hlstd{}\hlopt{));}\\
\hllin{054\ }\hlstd{}\hlstd{\ \ \ \ }\hlstd{dataSource}\hlopt{.}\hlstd{}\hlkwd{setMaxConnectionsPerPartition}\hlstd{}\hlopt{(}\hlstd{Integer\\
\hllin{055\ }}\hlstd{\ \ \ \ \ \ \ \ }\hlstd{}\hlopt{.}\hlstd{}\hlkwd{parseInt}\hlstd{}\hlopt{(}\hlstd{}\hlkwd{getProperty}\hlstd{}\hlopt{(}\hlstd{}\hlstr{"db.}\Righttorque\\
\hllin{056\ }\hlstr{}\hlstd{\ \ \ \ \ \ \ \ }\hlstr{maxConnectionsPerPartition"}\hlstd{}\hlopt{)));}\\
\hllin{057\ }\hlstd{}\hlstd{\ \ \ \ }\hlstd{dataSource}\hlopt{.}\hlstd{}\hlkwd{setMinConnectionsPerPartition}\hlstd{}\hlopt{(}\hlstd{Integer\\
\hllin{058\ }}\hlstd{\ \ \ \ \ \ \ \ }\hlstd{}\hlopt{.}\hlstd{}\hlkwd{parseInt}\hlstd{}\hlopt{(}\hlstd{}\hlkwd{getProperty}\hlstd{}\hlopt{(}\hlstd{}\hlstr{"db.}\Righttorque\\
\hllin{059\ }\hlstr{}\hlstd{\ \ \ \ \ \ \ \ }\hlstr{minConnectionsPerPartition"}\hlstd{}\hlopt{)));}\\
\hllin{060\ }\hlstd{}\hlstd{\ \ \ \ }\hlstd{dataSource}\hlopt{.}\hlstd{}\hlkwd{setPartitionCount}\hlstd{}\hlopt{(}\hlstd{Integer}\hlopt{.}\hlstd{}\hlkwd{parseInt}\hlstd{}\hlopt{(}\Righttorque\\
\hllin{061\ }\hlstd{}\hlstd{\ \ \ \ }\hlstd{}\hlkwd{getProperty}\hlstd{}\hlopt{(}\hlstd{}\hlstr{"db.partitionCount"}\hlstd{}\hlopt{)));}\\
\hllin{062\ }\hlstd{}\hlstd{\ \ \ \ }\hlstd{dataSource}\hlopt{.}\hlstd{}\hlkwd{setAcquireIncrement}\hlstd{}\hlopt{(}\hlstd{Integer}\hlopt{.}\hlstd{}\hlkwd{parseInt}\hlstd{}\hlopt{(}\Righttorque\\
\hllin{063\ }\hlstd{}\hlstd{\ \ \ \ }\hlstd{}\hlkwd{getProperty}\hlstd{}\hlopt{(}\hlstd{}\hlstr{"db.acquireIncrement"}\hlstd{}\hlopt{)));}\\
\hllin{064\ }\hlstd{}\hlstd{\ \ \ \ }\hlstd{dataSource}\hlopt{.}\hlstd{}\hlkwd{setStatementsCacheSize}\hlstd{}\hlopt{(}\hlstd{Integer}\hlopt{.}\hlstd{}\hlkwd{parseInt}\hlstd{}\hlopt{(}\Righttorque\\
\hllin{065\ }\hlstd{}\hlstd{\ \ \ \ }\hlstd{}\hlkwd{getProperty}\hlstd{}\hlopt{(}\hlstd{}\hlstr{"db.statementsCacheSize"}\hlstd{}\hlopt{)));}\\
\hllin{066\ }\hlstd{}\hlstd{\ \ \ \ }\hlstd{}\hlkwa{return\ }\hlstd{dataSource}\hlopt{;}\\
\hllin{067\ }\hlstd{}\hlstd{\ \ }\hlstd{}\hlopt{\}}\\
\hllin{068\ }\hlstd{\\
\hllin{069\ }}\hlstd{\ \ }\hlstd{}\hlcom{/{*}{*}}\\
\hllin{070\ }\hlcom{}\hlstd{\ \ \ }\hlcom{{*}\ Bean:\ session\ factory.}\\
\hllin{071\ }\hlcom{}\hlstd{\ \ \ }\hlcom{{*}}\\
\hllin{072\ }\hlcom{}\hlstd{\ \ \ }\hlcom{{*}\ @return\ sessionFactory\ bean}\\
\hllin{073\ }\hlcom{}\hlstd{\ \ \ }\hlcom{{*}/}\hlstd{\\
\hllin{074\ }}\hlstd{\ \ }\hlstd{}\hlslc{//\ Session\ Factory}\\
\hllin{075\ }\hlstd{}\hlstd{\ \ }\hlstd{}\hlkwc{@Bean}\hlstd{}\hlopt{(}\hlstd{name\ }\hlopt{=\ }\hlstd{}\hlstr{"sessionFactory"}\hlstd{}\hlopt{)}\\
\hllin{076\ }\hlstd{}\hlstd{\ \ }\hlstd{}\hlkwc{@Lazy}\hlstd{}\hlopt{(}\hlstd{false}\hlopt{)}\\
\hllin{077\ }\hlstd{}\hlstd{\ \ }\hlstd{}\hlkwa{public\ }\hlstd{LocalSessionFactoryBean\ }\hlkwd{sessionFactory}\hlstd{}\hlopt{()\ \{}\\
\hllin{078\ }\hlstd{}\hlstd{\ \ \ \ }\hlstd{LocalSessionFactoryBean\ localSessionFactoryBean\ }\hlopt{=\ }\hlstd{}\hlkwa{new\ }\Righttorque\\
\hllin{079\ }\hlstd{}\hlstd{\ \ \ \ }\hlstd{}\hlkwd{LocalSessionFactoryBean}\hlstd{}\hlopt{();}\\
\hllin{080\ }\hlstd{\\
\hllin{081\ }}\hlstd{\ \ \ \ }\hlstd{localSessionFactoryBean}\hlopt{.}\hlstd{}\hlkwd{setDataSource}\hlstd{}\hlopt{(}\hlstd{}\hlkwa{this}\hlstd{}\hlopt{.}\hlstd{}\hlkwd{dataSource}\hlstd{}\hlopt{())}\Righttorque\\
\hllin{082\ }\hlstd{}\hlstd{\ \ \ \ }\hlstd{}\hlopt{;}\\
\hllin{083\ }\hlstd{}\hlstd{\ \ \ \ }\hlstd{localSessionFactoryBean}\hlopt{.}\hlstd{}\hlkwd{setHibernateProperties}\hlstd{}\hlopt{(}\hlstd{}\hlkwa{this}\hlstd{}\hlopt{.}\Righttorque\\
\hllin{084\ }\hlstd{}\hlstd{\ \ \ \ }\hlstd{}\hlkwd{getHibernateProperties}\hlstd{}\hlopt{());}\\
\hllin{085\ }\hlstd{}\hlstd{\ \ \ \ }\hlstd{localSessionFactoryBean}\hlopt{.}\hlstd{}\hlkwd{setAnnotatedClasses}\hlstd{}\hlopt{(}\hlstd{Article}\hlopt{.}\Righttorque\\
\hllin{086\ }\hlstd{}\hlstd{\ \ \ \ }\hlstd{}\hlkwa{class}\hlstd{}\hlopt{,}\\
\hllin{087\ }\hlstd{}\hlstd{\ \ \ \ \ \ \ \ }\hlstd{Code}\hlopt{.}\hlstd{}\hlkwa{class}\hlstd{}\hlopt{,}\\
\hllin{088\ }\hlstd{}\hlstd{\ \ \ \ \ \ \ \ }\hlstd{CompileInfo}\hlopt{.}\hlstd{}\hlkwa{class}\hlstd{}\hlopt{,}\\
\hllin{089\ }\hlstd{}\hlstd{\ \ \ \ \ \ \ \ }\hlstd{Contest}\hlopt{.}\hlstd{}\hlkwa{class}\hlstd{}\hlopt{,}\\
\hllin{090\ }\hlstd{}\hlstd{\ \ \ \ \ \ \ \ }\hlstd{ContestProblem}\hlopt{.}\hlstd{}\hlkwa{class}\hlstd{}\hlopt{,}\\
\hllin{091\ }\hlstd{}\hlstd{\ \ \ \ \ \ \ \ }\hlstd{ContestTeamInfo}\hlopt{.}\hlstd{}\hlkwa{class}\hlstd{}\hlopt{,}\\
\hllin{092\ }\hlstd{}\hlstd{\ \ \ \ \ \ \ \ }\hlstd{ContestUser}\hlopt{.}\hlstd{}\hlkwa{class}\hlstd{}\hlopt{,}\\
\hllin{093\ }\hlstd{}\hlstd{\ \ \ \ \ \ \ \ }\hlstd{Department}\hlopt{.}\hlstd{}\hlkwa{class}\hlstd{}\hlopt{,}\\
\hllin{094\ }\hlstd{}\hlstd{\ \ \ \ \ \ \ \ }\hlstd{Language}\hlopt{.}\hlstd{}\hlkwa{class}\hlstd{}\hlopt{,}\\
\hllin{095\ }\hlstd{}\hlstd{\ \ \ \ \ \ \ \ }\hlstd{Message}\hlopt{.}\hlstd{}\hlkwa{class}\hlstd{}\hlopt{,}\\
\hllin{096\ }\hlstd{}\hlstd{\ \ \ \ \ \ \ \ }\hlstd{Problem}\hlopt{.}\hlstd{}\hlkwa{class}\hlstd{}\hlopt{,}\\
\hllin{097\ }\hlstd{}\hlstd{\ \ \ \ \ \ \ \ }\hlstd{ProblemTag}\hlopt{.}\hlstd{}\hlkwa{class}\hlstd{}\hlopt{,}\\
\hllin{098\ }\hlstd{}\hlstd{\ \ \ \ \ \ \ \ }\hlstd{Status}\hlopt{.}\hlstd{}\hlkwa{class}\hlstd{}\hlopt{,}\\
\hllin{099\ }\hlstd{}\hlstd{\ \ \ \ \ \ \ \ }\hlstd{Tag}\hlopt{.}\hlstd{}\hlkwa{class}\hlstd{}\hlopt{,}\\
\hllin{100\ }\hlstd{}\hlstd{\ \ \ \ \ \ \ \ }\hlstd{TrainingContest}\hlopt{.}\hlstd{}\hlkwa{class}\hlstd{}\hlopt{,}\\
\hllin{101\ }\hlstd{}\hlstd{\ \ \ \ \ \ \ \ }\hlstd{TrainingStatus}\hlopt{.}\hlstd{}\hlkwa{class}\hlstd{}\hlopt{,}\\
\hllin{102\ }\hlstd{}\hlstd{\ \ \ \ \ \ \ \ }\hlstd{TrainingUser}\hlopt{.}\hlstd{}\hlkwa{class}\hlstd{}\hlopt{,}\\
\hllin{103\ }\hlstd{}\hlstd{\ \ \ \ \ \ \ \ }\hlstd{User}\hlopt{.}\hlstd{}\hlkwa{class}\hlstd{}\hlopt{,}\\
\hllin{104\ }\hlstd{}\hlstd{\ \ \ \ \ \ \ \ }\hlstd{UserSerialKey}\hlopt{.}\hlstd{}\hlkwa{class}\hlstd{}\hlopt{);}\\
\hllin{105\ }\hlstd{\\
\hllin{106\ }}\hlstd{\ \ \ \ }\hlstd{}\hlkwa{return\ }\hlstd{localSessionFactoryBean}\hlopt{;}\\
\hllin{107\ }\hlstd{}\hlstd{\ \ }\hlstd{}\hlopt{\}}\\
\hllin{108\ }\hlstd{\\
\hllin{109\ }}\hlstd{\ \ }\hlstd{}\hlcom{/{*}{*}}\\
\hllin{110\ }\hlcom{}\hlstd{\ \ \ }\hlcom{{*}\ Bean:\ transaction\ manager.}\\
\hllin{111\ }\hlcom{}\hlstd{\ \ \ }\hlcom{{*}}\\
\hllin{112\ }\hlcom{}\hlstd{\ \ \ }\hlcom{{*}\ @return\ transactionManagerBean}\\
\hllin{113\ }\hlcom{}\hlstd{\ \ \ }\hlcom{{*}/}\hlstd{\\
\hllin{114\ }}\hlstd{\ \ }\hlstd{}\hlslc{//\ 事务管理工具}\\
\hllin{115\ }\hlstd{}\hlstd{\ \ }\hlstd{}\hlkwc{@Bean}\hlstd{}\hlopt{(}\hlstd{name\ }\hlopt{=\ }\hlstd{}\hlstr{"transactionManager"}\hlstd{}\hlopt{)}\\
\hllin{116\ }\hlstd{}\hlstd{\ \ }\hlstd{}\hlkwa{public\ }\hlstd{HibernateTransactionManager\ }\hlkwd{transactionManager}\hlstd{}\hlopt{()\ \{}\\
\hllin{117\ }\hlstd{}\hlstd{\ \ \ \ }\hlstd{HibernateTransactionManager\ transactionManager\ }\hlopt{=\ }\hlstd{}\hlkwa{new\ }\Righttorque\\
\hllin{118\ }\hlstd{}\hlstd{\ \ \ \ }\hlstd{}\hlkwd{HibernateTransactionManager}\hlstd{}\hlopt{();}\\
\hllin{119\ }\hlstd{}\hlstd{\ \ \ \ }\hlstd{transactionManager}\hlopt{.}\hlstd{}\hlkwd{setSessionFactory}\hlstd{}\hlopt{(}\hlstd{}\hlkwa{this}\hlstd{}\hlopt{.}\Righttorque\\
\hllin{120\ }\hlstd{}\hlstd{\ \ \ \ }\hlstd{}\hlkwd{sessionFactory}\hlstd{}\hlopt{().}\hlstd{}\hlkwd{getObject}\hlstd{}\hlopt{());}\\
\hllin{121\ }\hlstd{}\hlstd{\ \ \ \ }\hlstd{}\hlkwa{return\ }\hlstd{transactionManager}\hlopt{;}\\
\hllin{122\ }\hlstd{}\hlstd{\ \ }\hlstd{}\hlopt{\}}\\
\hllin{123\ }\hlstd{\\
\hllin{124\ }}\hlstd{\ \ }\hlstd{}\hlcom{/{*}{*}}\\
\hllin{125\ }\hlcom{}\hlstd{\ \ \ }\hlcom{{*}\ Hibernate\ properties.}\\
\hllin{126\ }\hlcom{}\hlstd{\ \ \ }\hlcom{{*}}\\
\hllin{127\ }\hlcom{}\hlstd{\ \ \ }\hlcom{{*}\ @return\ properties}\\
\hllin{128\ }\hlcom{}\hlstd{\ \ \ }\hlcom{{*}/}\hlstd{\\
\hllin{129\ }}\hlstd{\ \ }\hlstd{}\hlslc{//\ Hibernate配置}\\
\hllin{130\ }\hlstd{}\hlstd{\ \ }\hlstd{}\hlkwa{private\ }\hlstd{Properties\ }\hlkwd{getHibernateProperties}\hlstd{}\hlopt{()\ \{}\\
\hllin{131\ }\hlstd{}\hlstd{\ \ \ \ }\hlstd{Properties\ properties\ }\hlopt{=\ }\hlstd{}\hlkwa{new\ }\hlstd{}\hlkwd{Properties}\hlstd{}\hlopt{();}\\
\hllin{132\ }\hlstd{}\hlstd{\ \ \ \ }\hlstd{properties}\hlopt{.}\hlstd{}\hlkwd{setProperty}\hlstd{}\hlopt{(}\hlstd{}\hlstr{"hibernate.dialect"}\hlstd{}\hlopt{,\ }\hlstd{environment}\hlopt{.}\Righttorque\\
\hllin{133\ }\hlstd{}\hlstd{\ \ \ \ }\hlstd{}\hlkwd{getProperty}\hlstd{}\hlopt{(}\hlstd{}\hlstr{"hibernate.dialect"}\hlstd{}\hlopt{));}\\
\hllin{134\ }\hlstd{}\hlstd{\ \ \ \ }\hlstd{properties}\hlopt{.}\hlstd{}\hlkwd{setProperty}\hlstd{}\hlopt{(}\hlstd{}\hlstr{"hibernate.show\textunderscore sql"}\hlstd{}\hlopt{,\ }\Righttorque\\
\hllin{135\ }\hlstd{}\hlstd{\ \ \ \ }\hlstd{environment}\hlopt{.}\hlstd{}\hlkwd{getProperty}\hlstd{}\hlopt{(}\hlstd{}\hlstr{"hibernate.show\textunderscore sql"}\hlstd{}\hlopt{));}\\
\hllin{136\ }\hlstd{}\hlstd{\ \ \ \ }\hlstd{properties}\hlopt{.}\hlstd{}\hlkwd{setProperty}\hlstd{}\hlopt{(}\hlstd{}\hlstr{"hibernate.format"}\hlstd{}\hlopt{,\ }\hlstd{environment}\hlopt{.}\Righttorque\\
\hllin{137\ }\hlstd{}\hlstd{\ \ \ \ }\hlstd{}\hlkwd{getProperty}\hlstd{}\hlopt{(}\hlstd{}\hlstr{"hibernate.format\textunderscore sql"}\hlstd{}\hlopt{));}\\
\hllin{138\ }\hlstd{}\hlstd{\ \ \ \ }\hlstd{properties}\hlopt{.}\hlstd{}\hlkwd{setProperty}\hlstd{}\hlopt{(}\hlstd{}\hlstr{"hibernate.}\Righttorque\\
\hllin{139\ }\hlstr{}\hlstd{\ \ \ \ }\hlstr{current\textunderscore session\textunderscore context\textunderscore class"}\hlstd{}\hlopt{,}\\
\hllin{140\ }\hlstd{}\hlstd{\ \ \ \ \ \ \ \ }\hlstd{environment}\hlopt{.}\hlstd{}\hlkwd{getProperty}\hlstd{}\hlopt{(}\hlstd{}\hlstr{"hibernate.}\Righttorque\\
\hllin{141\ }\hlstr{}\hlstd{\ \ \ \ \ \ \ \ }\hlstr{current\textunderscore session\textunderscore context\textunderscore class"}\hlstd{}\hlopt{));}\\
\hllin{142\ }\hlstd{}\hlstd{\ \ \ \ }\hlstd{}\hlkwa{return\ }\hlstd{properties}\hlopt{;}\\
\hllin{143\ }\hlstd{}\hlstd{\ \ }\hlstd{}\hlopt{\}}\\
\hllin{144\ }\hlstd{\\
\hllin{145\ }}\hlstd{\ \ }\hlstd{}\hlcom{/{*}{*}}\\
\hllin{146\ }\hlcom{}\hlstd{\ \ \ }\hlcom{{*}\ Simply\ get\ property\ in\ PropertySource.}\\
\hllin{147\ }\hlcom{}\hlstd{\ \ \ }\hlcom{{*}}\\
\hllin{148\ }\hlcom{}\hlstd{\ \ \ }\hlcom{{*}\ @param\ name\ property\ name}\\
\hllin{149\ }\hlcom{}\hlstd{\ \ \ }\hlcom{{*}\ @return\ property\ value}\\
\hllin{150\ }\hlcom{}\hlstd{\ \ \ }\hlcom{{*}/}\hlstd{\\
\hllin{151\ }}\hlstd{\ \ }\hlstd{}\hlkwa{private\ }\hlstd{String\ }\hlkwd{getProperty}\hlstd{}\hlopt{(}\hlstd{}\hlkwa{final\ }\hlstd{String\ name}\hlopt{)\ \{}\\
\hllin{152\ }\hlstd{}\hlstd{\ \ \ \ }\hlstd{}\hlkwa{return\ }\hlstd{environment}\hlopt{.}\hlstd{}\hlkwd{getProperty}\hlstd{}\hlopt{(}\hlstd{name}\hlopt{);}\\
\hllin{153\ }\hlstd{}\hlstd{\ \ }\hlstd{}\hlopt{\}}\\
\hllin{154\ }\hlstd{}\hlopt{\}}\hlstd{}\\
\mbox{}
\normalfont
\normalsize


系统启动时默认将resources.properties文件中的键值对初始化成一个Environment实例,我们可以通过getProperty(String): String方法来获得对应的值。

下面是\textbf{resources.properties}文件:

\noindent
\ttfamily
\hlstd{}\hllin{001\ }\hlslc{\#数据驱动}\\
\hllin{002\ }\hlstd{}\hlkwb{db.driver}\hlstd{}\hlopt{=}\hlstd{}\hlkwc{com.mysql.jdbc.Driver}\\
\hllin{003\ }\hlstd{}\hlslc{\#数据库连接地址}\\
\hllin{004\ }\hlstd{}\hlkwb{db.url}\hlstd{}\hlopt{=}\hlstd{}\hlkwc{jdbc:mysql://localhost:}\Righttorque\\
\hllin{005\ }\hlstd{}\hlkwb{3306/uestcoj?useUnicode}\hlstd{}\hlopt{=}\hlstd{}\hlkwc{true\&characterEncoding=UTF{-}8}\\
\hllin{006\ }\hlstd{}\hlslc{\#数据库用户名及密码}\\
\hllin{007\ }\hlstd{}\hlkwb{db.username}\hlstd{}\hlopt{=}\hlstd{}\hlkwc{root}\\
\hllin{008\ }\hlstd{}\hlkwb{db.password}\hlstd{}\hlopt{=}\hlstd{}\hlkwc{root}\\
\hllin{009\ }\hlstd{}\\
\hllin{010\ }\hlslc{\#以下是数据库性能配置}\\
\hllin{011\ }\hlstd{}\hlkwb{db.maxConnectionsPerPartition}\hlstd{}\hlopt{=}\hlstd{}\hlkwc{30}\\
\hllin{012\ }\hlstd{}\hlkwb{db.minConnectionsPerPartition}\hlstd{}\hlopt{=}\hlstd{}\hlkwc{10}\\
\hllin{013\ }\hlstd{}\hlkwb{db.partitionCount}\hlstd{}\hlopt{=}\hlstd{}\hlkwc{3}\\
\hllin{014\ }\hlstd{}\hlkwb{db.acquireIncrement}\hlstd{}\hlopt{=}\hlstd{}\hlkwc{5}\\
\hllin{015\ }\hlstd{}\hlkwb{db.statementsCacheSize}\hlstd{}\hlopt{=}\hlstd{}\hlkwc{100}\\
\hllin{016\ }\hlstd{}\hlkwb{db.idleMaxAge}\hlstd{}\hlopt{=}\hlstd{}\hlkwc{5}\\
\hllin{017\ }\hlstd{}\hlkwb{db.idleConnectionTestPeriod}\hlstd{}\hlopt{=}\hlstd{}\hlkwc{5}\\
\hllin{018\ }\hlstd{}\\
\hllin{019\ }\hlslc{\#Hibernate配置}\\
\hllin{020\ }\hlstd{}\hlkwb{hibernate.dialect}\hlstd{}\hlopt{=}\hlstd{}\hlkwc{org.hibernate.dialect.MySQL5Dialect}\\
\hllin{021\ }\hlstd{}\hlkwb{hibernate.show\textunderscore sql}\hlstd{}\hlopt{=}\hlstd{}\hlkwc{false}\\
\hllin{022\ }\hlstd{}\hlkwb{hibernate.format\textunderscore sql}\hlstd{}\hlopt{=}\hlstd{}\hlkwc{false}\\
\hllin{023\ }\hlstd{}\hlkwb{hibernate.current\textunderscore session\textunderscore context\textunderscore class}\hlstd{}\hlopt{=}\hlstd{}\hlkwc{org.springframework.}\Righttorque\\
\hllin{024\ }\hlstd{orm.hibernate4.SpringSessionContext}\\
\hllin{025\ }\\
\hllin{026\ }\hlkwb{sessionFactory.annotatedPackages}\hlstd{}\hlopt{=}\hlstd{}\hlkwc{cn.edu.uestc.acmicpc.db.}\Righttorque\\
\hllin{027\ }\hlstd{entity}\\
\mbox{}
\normalfont
\normalsize


有了\textbf{Environment}实例,我们就可以将配置信息从代码中分离开来。

\subsubsection{WebMVCResource}
为了方便进行测试,我们将一些比较特殊的资源从WebMVCConfig.java中独立开来,放到WebMVCResource.java中。这里主要做了两件事情:一个是配置视图解析器,在这里我们设置视图地址的前缀和后缀,方便Controller调用视图。另外一件事就是配置了JSON数据的转换器,用于解析和构建JSON数据,这里我们使用了fastjson\footnote{Fastjson是一个Java语言编写的JSON处理器,由阿里巴巴公司开发。}。

\noindent
\ttfamily
\hlstd{}\hllin{001\ }\hlcom{/{*}{*}}\\
\hllin{002\ }\hlcom{\ {*}\ Spring\ MVC\ Configuration\ file\ resource.}\\
\hllin{003\ }\hlcom{\ {*}/}\hlstd{}\\
\hllin{004\ }\hlkwa{public\ class\ }\hlstd{WebMVCResource\ }\hlopt{\{}\\
\hllin{005\ }\hlstd{\\
\hllin{006\ }}\hlstd{\ \ }\hlstd{}\hlslc{//\ 视图解析器配置}\\
\hllin{007\ }\hlstd{}\hlstd{\ \ }\hlstd{}\hlkwa{public\ static\ }\hlstd{ViewResolver\ }\hlkwd{viewResolver}\hlstd{}\hlopt{()\ \{}\\
\hllin{008\ }\hlstd{}\hlstd{\ \ \ \ }\hlstd{InternalResourceViewResolver\ viewResolver\ }\hlopt{=\ }\hlstd{}\hlkwa{new\ }\Righttorque\\
\hllin{009\ }\hlstd{}\hlstd{\ \ \ \ }\hlstd{}\hlkwd{InternalResourceViewResolver}\hlstd{}\hlopt{();}\\
\hllin{010\ }\hlstd{\\
\hllin{011\ }}\hlstd{\ \ \ \ }\hlstd{viewResolver}\hlopt{.}\hlstd{}\hlkwd{setPrefix}\hlstd{}\hlopt{(}\hlstd{}\hlstr{"/WEB{-}INF/views/"}\hlstd{}\hlopt{);}\\
\hllin{012\ }\hlstd{}\hlstd{\ \ \ \ }\hlstd{viewResolver}\hlopt{.}\hlstd{}\hlkwd{setSuffix}\hlstd{}\hlopt{(}\hlstd{}\hlstr{".jsp"}\hlstd{}\hlopt{);}\\
\hllin{013\ }\hlstd{\\
\hllin{014\ }}\hlstd{\ \ \ \ }\hlstd{}\hlkwa{return\ }\hlstd{viewResolver}\hlopt{;}\\
\hllin{015\ }\hlstd{}\hlstd{\ \ }\hlstd{}\hlopt{\}}\\
\hllin{016\ }\hlstd{\\
\hllin{017\ }}\hlstd{\ \ }\hlstd{}\hlslc{//\ 设置FastJson为默认的JSON格式转换器}\\
\hllin{018\ }\hlstd{}\hlstd{\ \ }\hlstd{}\hlkwa{public\ static\ }\hlstd{HttpMessageConverter}\hlopt{$<$}\hlstd{?}\hlopt{$>${[}{]}\ }\hlstd{}\hlkwd{messageConverters}\hlstd{}\hlopt{(}\Righttorque\\
\hllin{019\ }\hlstd{}\hlstd{\ \ }\hlstd{}\hlopt{)\ \{}\\
\hllin{020\ }\hlstd{}\hlstd{\ \ \ \ }\hlstd{HttpMessageConverter}\hlopt{$<$}\hlstd{?}\hlopt{$>${[}{]}\ }\hlstd{converters\ }\hlopt{=\ }\hlstd{}\hlkwa{new\ }\Righttorque\\
\hllin{021\ }\hlstd{}\hlstd{\ \ \ \ }\hlstd{HttpMessageConverter}\hlopt{$<$}\hlstd{?}\hlopt{$>${[}}\hlstd{}\hlnum{1}\hlstd{}\hlopt{{]};}\\
\hllin{022\ }\hlstd{}\hlstd{\ \ \ \ }\hlstd{FastJsonHttpMessageConverter\ \Righttorque\\
\hllin{023\ }}\hlstd{\ \ \ \ }\hlstd{fastJsonHttpMessageConverter\ }\hlopt{=\ }\hlstd{}\hlkwa{new\ }\Righttorque\\
\hllin{024\ }\hlstd{}\hlstd{\ \ \ \ }\hlstd{}\hlkwd{FastJsonHttpMessageConverter}\hlstd{}\hlopt{();}\\
\hllin{025\ }\hlstd{}\hlstd{\ \ \ \ }\hlstd{List}\hlopt{$<$}\hlstd{MediaType}\hlopt{$>$\ }\hlstd{mediaTypes\ }\hlopt{=\ }\hlstd{}\hlkwa{new\ }\hlstd{LinkedList}\hlopt{$<$$>$();}\\
\hllin{026\ }\hlstd{}\hlstd{\ \ \ \ }\hlstd{mediaTypes}\hlopt{.}\hlstd{}\hlkwd{add}\hlstd{}\hlopt{(}\hlstd{MediaType}\hlopt{.}\hlstd{APPLICATION\textunderscore JSON}\hlopt{);}\\
\hllin{027\ }\hlstd{}\hlstd{\ \ \ \ }\hlstd{fastJsonHttpMessageConverter}\hlopt{.}\hlstd{}\hlkwd{setSupportedMediaTypes}\hlstd{}\hlopt{(}\Righttorque\\
\hllin{028\ }\hlstd{}\hlstd{\ \ \ \ }\hlstd{mediaTypes}\hlopt{);}\\
\hllin{029\ }\hlstd{}\hlstd{\ \ \ \ }\hlstd{converters}\hlopt{{[}}\hlstd{}\hlnum{0}\hlstd{}\hlopt{{]}\ =\ }\hlstd{fastJsonHttpMessageConverter}\hlopt{;}\\
\hllin{030\ }\hlstd{}\hlstd{\ \ \ \ }\hlstd{}\hlkwa{return\ }\hlstd{converters}\hlopt{;}\\
\hllin{031\ }\hlstd{}\hlstd{\ \ }\hlstd{}\hlopt{\}}\\
\hllin{032\ }\hlstd{}\\
\hllin{033\ }\hlopt{\}}\hlstd{}\\
\mbox{}
\normalfont
\normalsize


\subsubsection{WebMVCConfig}
这个文件是SpringMVC框架的配置文件,与之前的ApplicationContextConfig类似,这里配置了与Web相关的参数。

\noindent
\ttfamily
\hlstd{}\hllin{001\ }\hlcom{/{*}{*}}\\
\hllin{002\ }\hlcom{\ {*}\ Spring\ MVC\ configuration\ file.}\\
\hllin{003\ }\hlcom{\ {*}/}\hlstd{}\\
\hllin{004\ }\hlkwc{@Configuration}\\
\hllin{005\ }\hlstd{}\hlslc{//\ 开启Web\ MVC模式}\\
\hllin{006\ }\hlstd{}\hlkwc{@EnableWebMvc}\\
\hllin{007\ }\hlstd{}\hlslc{//\ 指定Controller包位置}\\
\hllin{008\ }\hlstd{}\hlkwc{@ComponentScan}\hlstd{}\hlopt{(}\hlstd{basePackages\ }\hlopt{=\ \{}\\
\hllin{009\ }\hlstd{}\hlstd{\ \ \ \ }\hlstd{}\hlstr{"cn.edu.uestc.acmicpc.web.oj.controller"}\hlstd{}\\
\hllin{010\ }\hlopt{\})}\\
\hllin{011\ }\hlstd{}\hlslc{//\ 开启AspectJ代理}\\
\hllin{012\ }\hlstd{}\hlkwc{@EnableAspectJAutoProxy}\hlstd{}\hlopt{(}\hlstd{proxyTargetClass\ }\hlopt{=\ }\hlstd{true}\hlopt{)}\\
\hllin{013\ }\hlstd{}\hlkwa{public\ class\ }\hlstd{WebMVCConfig\ }\hlkwa{extends\ }\hlstd{WebMvcConfigurerAdapter\ }\hlopt{\{}\\
\hllin{014\ }\hlstd{\\
\hllin{015\ }}\hlstd{\ \ }\hlstd{}\hlslc{//\ 注册静态资源代理}\\
\hllin{016\ }\hlstd{}\hlstd{\ \ }\hlstd{}\hlkwc{@Override}\\
\hllin{017\ }\hlstd{}\hlstd{\ \ }\hlstd{}\hlkwa{public\ }\hlstd{}\hlkwb{void\ }\hlstd{}\hlkwd{addResourceHandlers}\hlstd{}\hlopt{(}\hlstd{ResourceHandlerRegistry\ \Righttorque\\
\hllin{018\ }}\hlstd{\ \ }\hlstd{registry}\hlopt{)\ \{}\\
\hllin{019\ }\hlstd{}\hlstd{\ \ \ \ }\hlstd{registry}\hlopt{.}\hlstd{}\hlkwd{addResourceHandler}\hlstd{}\hlopt{(}\hlstd{}\hlstr{"/images/{*}{*}"}\hlstd{}\hlopt{).}\Righttorque\\
\hllin{020\ }\hlstd{}\hlstd{\ \ \ \ }\hlstd{}\hlkwd{addResourceLocations}\hlstd{}\hlopt{(}\hlstd{}\hlstr{"/images/{*}{*}"}\hlstd{}\hlopt{);}\\
\hllin{021\ }\hlstd{}\hlstd{\ \ \ \ }\hlstd{registry}\hlopt{.}\hlstd{}\hlkwd{addResourceHandler}\hlstd{}\hlopt{(}\hlstd{}\hlstr{"/plugins/{*}{*}"}\hlstd{}\hlopt{).}\Righttorque\\
\hllin{022\ }\hlstd{}\hlstd{\ \ \ \ }\hlstd{}\hlkwd{addResourceLocations}\hlstd{}\hlopt{(}\hlstd{}\hlstr{"/plugins/{*}{*}"}\hlstd{}\hlopt{);}\\
\hllin{023\ }\hlstd{}\hlstd{\ \ \ \ }\hlstd{registry}\hlopt{.}\hlstd{}\hlkwd{addResourceHandler}\hlstd{}\hlopt{(}\hlstd{}\hlstr{"/scripts/{*}{*}"}\hlstd{}\hlopt{).}\Righttorque\\
\hllin{024\ }\hlstd{}\hlstd{\ \ \ \ }\hlstd{}\hlkwd{addResourceLocations}\hlstd{}\hlopt{(}\hlstd{}\hlstr{"/scripts/{*}{*}"}\hlstd{}\hlopt{);}\\
\hllin{025\ }\hlstd{}\hlstd{\ \ \ \ }\hlstd{registry}\hlopt{.}\hlstd{}\hlkwd{addResourceHandler}\hlstd{}\hlopt{(}\hlstd{}\hlstr{"/styles/{*}{*}"}\hlstd{}\hlopt{).}\Righttorque\\
\hllin{026\ }\hlstd{}\hlstd{\ \ \ \ }\hlstd{}\hlkwd{addResourceLocations}\hlstd{}\hlopt{(}\hlstd{}\hlstr{"/styles/{*}{*}"}\hlstd{}\hlopt{);}\\
\hllin{027\ }\hlstd{}\hlstd{\ \ \ \ }\hlstd{registry}\hlopt{.}\hlstd{}\hlkwd{addResourceHandler}\hlstd{}\hlopt{(}\hlstd{}\hlstr{"/font/{*}{*}"}\hlstd{}\hlopt{).}\Righttorque\\
\hllin{028\ }\hlstd{}\hlstd{\ \ \ \ }\hlstd{}\hlkwd{addResourceLocations}\hlstd{}\hlopt{(}\hlstd{}\hlstr{"/font/{*}{*}"}\hlstd{}\hlopt{);}\\
\hllin{029\ }\hlstd{}\hlstd{\ \ }\hlstd{}\hlopt{\}}\\
\hllin{030\ }\hlstd{\\
\hllin{031\ }}\hlstd{\ \ }\hlstd{}\hlslc{//\ 开启ServletHandler}\\
\hllin{032\ }\hlstd{}\hlstd{\ \ }\hlstd{}\hlkwc{@Override}\\
\hllin{033\ }\hlstd{}\hlstd{\ \ }\hlstd{}\hlkwa{public\ }\hlstd{}\hlkwb{void\ }\hlstd{}\hlkwd{configureDefaultServletHandling}\hlstd{}\hlopt{(}\\
\hllin{034\ }\hlstd{}\hlstd{\ \ \ \ \ \ }\hlstd{DefaultServletHandlerConfigurer\ \Righttorque\\
\hllin{035\ }}\hlstd{\ \ \ \ \ \ }\hlstd{defaultServletHandlerConfigurer}\hlopt{)\ \{}\\
\hllin{036\ }\hlstd{}\hlstd{\ \ \ \ }\hlstd{defaultServletHandlerConfigurer}\hlopt{.}\hlstd{}\hlkwd{enable}\hlstd{}\hlopt{();}\\
\hllin{037\ }\hlstd{}\hlstd{\ \ }\hlstd{}\hlopt{\}}\\
\hllin{038\ }\hlstd{\\
\hllin{039\ }}\hlstd{\ \ }\hlstd{}\hlslc{//\ 注册消息转换器}\\
\hllin{040\ }\hlstd{}\hlstd{\ \ }\hlstd{}\hlkwc{@Override}\\
\hllin{041\ }\hlstd{}\hlstd{\ \ }\hlstd{}\hlkwa{public\ }\hlstd{}\hlkwb{void\ }\hlstd{}\hlkwd{configureMessageConverters}\hlstd{}\hlopt{(}\hlstd{List}\hlopt{$<$}\Righttorque\\
\hllin{042\ }\hlstd{}\hlstd{\ \ }\hlstd{HttpMessageConverter}\hlopt{$<$}\hlstd{?}\hlopt{$>$$>$\ }\hlstd{converters}\hlopt{)\ \{}\\
\hllin{043\ }\hlstd{}\hlstd{\ \ \ \ }\hlstd{converters}\hlopt{.}\hlstd{}\hlkwd{addAll}\hlstd{}\hlopt{(}\hlstd{Arrays}\hlopt{.}\hlstd{}\hlkwd{asList}\hlstd{}\hlopt{(}\hlstd{WebMVCResource}\hlopt{.}\Righttorque\\
\hllin{044\ }\hlstd{}\hlstd{\ \ \ \ }\hlstd{}\hlkwd{messageConverters}\hlstd{}\hlopt{()));}\\
\hllin{045\ }\hlstd{}\hlstd{\ \ }\hlstd{}\hlopt{\}}\\
\hllin{046\ }\hlstd{\\
\hllin{047\ }}\hlstd{\ \ }\hlstd{}\hlslc{//\ 视图解析器}\\
\hllin{048\ }\hlstd{}\hlstd{\ \ }\hlstd{}\hlkwc{@Bean}\\
\hllin{049\ }\hlstd{}\hlstd{\ \ }\hlstd{}\hlkwa{public\ }\hlstd{ViewResolver\ }\hlkwd{viewResolver}\hlstd{}\hlopt{()\ \{}\\
\hllin{050\ }\hlstd{}\hlstd{\ \ \ \ }\hlstd{}\hlkwa{return\ }\hlstd{WebMVCResource}\hlopt{.}\hlstd{}\hlkwd{viewResolver}\hlstd{}\hlopt{();}\\
\hllin{051\ }\hlstd{}\hlstd{\ \ }\hlstd{}\hlopt{\}}\\
\hllin{052\ }\hlstd{\\
\hllin{053\ }}\hlstd{\ \ }\hlstd{}\hlslc{//\ 启用多文件上传}\\
\hllin{054\ }\hlstd{}\hlstd{\ \ }\hlstd{}\hlkwc{@Bean}\\
\hllin{055\ }\hlstd{}\hlstd{\ \ }\hlstd{}\hlkwa{public\ }\hlstd{MultipartResolver\ }\hlkwd{multipartResolver}\hlstd{}\hlopt{()\ \{}\\
\hllin{056\ }\hlstd{}\hlstd{\ \ \ \ }\hlstd{}\hlkwa{return\ new\ }\hlstd{}\hlkwd{CommonsMultipartResolver}\hlstd{}\hlopt{();}\\
\hllin{057\ }\hlstd{}\hlstd{\ \ }\hlstd{}\hlopt{\}}\\
\hllin{058\ }\hlstd{}\\
\hllin{059\ }\hlopt{\}}\hlstd{}\\
\mbox{}
\normalfont
\normalsize


\subsection{Database Package详细设计}
\pic[htbp]{Database Package包图}{}{DatabasePackage}

这个包在MVC模型中处于Model层,所有与数据库有关的API都被包含在里面。

\subsubsection{Entity}
\pic[htbp]{Entity Package类图}{}{EntityPackage}

Entity即为实体,对应着MVC模型中的Model,它和数据库中的内容有着直接的一对一映射关系。本系统数据库较为复杂,这里不赘述每个实体的定义,我们来简述一下各个实体的作用,如表\ref{entitytable}所示

\threelinetable[htbp]{entitytable}{\textwidth}{ll}{Entity表}
{Entity & 作用\\
}{
Article & 文章的内容和基本信息\\
Code & 用户提交的代码\\
CompileInfo & 代码的编译信息\\
Contest & 比赛的基本信息\\
ContestProblem & 比赛和题目的对应关系\\
ContestTeamInfo & 参赛队伍的信息\\
ContestUser & 比赛的注册用户\\
Department & 学校的部门信息\\
Language & 可以使用的语言以及部分参数\\
}{
}

\subsubsection{Condition}
\pic[htbp]{Condition Package类图}{}{ConditionPackage}

我们在本系统中使用Hibernate作为持久层框架,它提供了强大的HQL查询语言,Condition包的主要功能就是提供了Condition组件,它可以翻译成HQL查询语言的where条件,来限定检索范围。

根据实际情况,本系统设计的Condition支持三种条件:
\begin{enumerate}
	\item Order条件:用来限定返回结果的顺序。
	\item PageInfo条件:用来实现返回结果的分页功能。
	\item 普通条件:既Entry,它既可以是一条普通的条件,如\textbf{userId = 5},也可以是一个Condition。在枚举类型ConditionType中,我们定义了许多常用的条件,如等于、不等于、小于、like、属于等等。
\end{enumerate}

对于每个数据库实体类型Entity,都有一个对应的Condition类,如Problem实体有对应的ProblemCondition。