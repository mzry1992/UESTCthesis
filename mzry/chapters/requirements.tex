% !Mode:: "TeX:UTF-8"

\chapter{系统需求分析}
\section{功能性需求分析}
\begin{itemize}
%	\item 对于评测器来说,本系统应该具有如下基本功能:
%\begin{enumerate}
%	\item 对于给定的程序源代码和对应的测试用例,评测器将源代码编译、运行并比较输出结果是否正确,同时将结果返回给用户。
%	\item 评测器可以对程序做出限制,基本的限制有时间限制,内存限制和权限限制,并且保证程序运行不会对系统造成破坏。
%	\item 不同的题目可以有不同的限制范围。
%\end{enumerate}
\item 对于用户来说,本系统应该具有如下功能:
\begin{enumerate}
	\item 注册并登陆到系统之中,并且每个用户都有自己的权限范围。
	\item 在权限范围内浏览题目、提交题目、浏览比赛、参与比赛、查看自己提交的代码状态、修改用户信息、发布帖子、留言。
\end{enumerate}
\item 对于管理员来说,本系统应具有如下功能:
\begin{description}
	\item[题目管理] 向题库中加入/管理题目,每个题目都拥有统一的格式,系统应该提供基本的编辑功能。
	\item[比赛管理] 向比赛列表中新建/管理比赛,并进行适当的统计操作。
	\item[提交管理] 查看用户提交的代码,在特殊情况下能够对提交进行Rejudge(重新测评)操作。
	\item[用户管理] 对用户的一些信息做出修改(提升权限,封禁用户)。
	\item[公告管理] 发布/修改公告来告知用户近期的比赛以及一些新闻。
	\item[系统维护] 对系统进行维护操作,如备份,回档。
\end{description}
\end{itemize}

\section{非功能性需求分析}
\subsection{界面需求}
易用性的高低决定着用户对产品的第一印象是好是坏。因为Online Judge需要向用户展示大量的数据,于是用户需要一个简洁、易懂的界面,这样他们就能不需要任何帮助地一眼获得需要的信息。考虑到实际情况,网站最低需要支持IE 7浏览器。

考虑到部分用户会使用移动设备登陆网站,该网站应当对移动设备有一定的优化。

\subsection{性能需求}
网站在各种操作和处理中响应速度应为秒级甚至毫秒级,达到实时要求。尽量减少卡顿情况。同时对于评测器来说要高效的评测代码,正常情况下不能出现待测试代码堆积的情况。同时作为一个成熟的系统应该能保证用户稳定地进行比赛、提交题目,程序在任何情况下都能正常运行,不出现崩溃的情况。

为了说明实际的使用情况,本文统计了老版本电子科技大学Online Judge的使用情况,如表\ref{onlinejudgeusage}所示:
\threelinetable[htbp]{onlinejudgeusage}{\textwidth}{llll}{电子科技大学Online Judge使用状况}
{项目 & 用户规模 & 提交规模 & 历时\\
}{
总体 & 32083 & 475583 & 4年\\
第11届电子科技大学校赛 & 94 & 1315 & 5小时\\
第五届ACM趣味程序设计竞赛第四场(正式赛)& 82 & 676 & 3小时\\
数学科学学院2013级C语言第六次上机练习 & 142 & 906 & 3小时15分钟\\
}{
}

基于上述统计,我们规定系统应该具备以下能力:
\begin{enumerate}
	\item 同时支持至少300个连接。
	\item 对大部分响应来说要求在1s内完成(考虑一般的网络连接状况)。
	\item 由于比赛中的提交具有突发性的特点,每分钟我们系统应该能够支持评测至少120秒程序。
\end{enumerate}

\subsection{维护需求}
一个软件难免需要人来维护,或者需要对功能进行升级。很多时候,维护或升级的人员因为原先的代码可读性差以致无法顺利地进行进一步操作。因此,系统的程序代码应该具备足够的易读性,具备标准的格式和简洁的代码风格,这样能使维护和升级顺利地进行。同时开发者还需提供详细的说明文档。

\subsection{运行环境需求}
这里的运行环境主要指用户端的环境,考虑到学校机房中的实际配置,我们有如下的最低要求:
\begin{enumerate}
	\item 支持IE 7+以及其余现代浏览器。
	\item 至少支持Windows XP系统。
	\item 能够在离线环境下使用。
\end{enumerate}