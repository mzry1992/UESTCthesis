% !Mode:: "TeX:UTF-8"

\chapter{引言}
\section{课题背景和意义}
ACM-ICPC(Association of Computing Machinery - ACM International Collegiate Programming Contest,美国计算机协会——国际大学生程序设计竞赛)是由国际计算机界历史悠久、颇具权威性的组织ACM于1970年发起组织的年度竞赛活动,是当今国际计算机界历史悠久并得到全球公认的规模最大、水平最高的国际大学生设计竞赛。大赛旨在展示大学生创新能力、团队精神和在压力下编写程序、分析和解决问题能力,迄今已经成功举办38届。比赛涌现出的优秀学生往往被各高校和许多知名企业所看重。

ACM-ICPC以团队的形式代表各学校参赛,每队由3名队员组成\footnote{每位队员必须是在校学生,有一定的年龄限制,并且最多可以参加2次全
球总决赛和5次区域选拔赛。}。比赛期间,每队使用1台电脑需要在5个小时内使用C、C++或Java中的一种编写程序解决7到10个问题。每个问题都有一组标准的测试数据以及对应的答案,选手程序完成之后提交裁判运行,裁判机运行选手提交的程序,通过其输出于标准答案想比较来得到结果,运行的结果会判定为``AC(正确)/WA(错误)/TLE(超时)/MLE(超出内存限制)/RE(运行错误)/PE(格式错误)''中的一种并及时通知参赛队。

电子科技大学从2005年起便开始参加这项竞赛,在最近的第38届ACM-ICPC亚洲区域赛中国大陆赛区共有成都、杭州、南京、长沙、长春5站,其中成都站的比赛由电子科技大学承办,本届比赛中,电子科技大学学子共获4金7银5铜。其中UESTC\_Aspidochelone代表队在成都站排名第二,在南京站获得亚军殊荣,顺利晋级2014年夏季在俄罗斯叶卡捷琳堡举行的世界总决赛\footnote{见:\url{http://www.new1.uestc.edu.cn/news/index/id/1056}}。

ACM-ICPC与其它竞赛最大的区别在于它采用的是机器评测的方法而不是依靠人的评价,它采用了黑盒测试\cite{beizer1995black}的思想来评判选手的程序。在黑盒测试中,测试者只知道程序的输入、输出和系统的功能,按照一定的规范设计出一系列测试案例来进行测试。在线程序评测系统(Online Judge)以此为基础,可以对多种语言的源代码进行自动编译、测试、分析及评判。除了被应用于程序设计竞赛,也有一些老师将其引入到日常的程序语言教学之中,并取得了很好的效果\cite{youfeng2009acm}\cite{guosongshan2007acm}。

\section{国内外研究现状}
目前已经存在许多不同种类的Online Judge,如表\ref{onlinejudges}所示:
\threelinetable[htbp]{onlinejudges}{\textwidth}{lll}{几个著名的评测网站}
{名称 & 来源 & 地址\\
}{
Topcoder & TopCoder, Inc & \url{http://community.topcoder.com/tc}\\
Codeforces & 萨拉托夫州立大学 & \url{http://codeforces.com/}\\
Project Euler & Colin Hughes & \url{https://projecteuler.net/}\\
HDOJ & 杭州电子科技大学 & \url{http://acm.hdu.edu.cn/}\\
POJ & 北京大学 & \url{http://poj.org/}\\
Virtual Judge & 华中科技大学 & \url{http://acm.hust.edu.cn/vjudge/toIndex.action}\\
ZOJ & 浙江大学 & \url{http://acm.zju.edu.cn/onlinejudge/}\\
}{}

其中HDOJ、POJ、ZOJ都属于传统的Online Judge,有着自己的题库和测评器。HDOJ如今已经成为了国内ACM竞赛界最为著名的Online Judge,每年暑假都会组织多校联合训练,平时还会承办各类程序设计竞赛(如腾讯编程马拉松)。ZOJ则是以每个月举行的浙大月赛而闻名。

相反,Virtual Judge没有自己的题库和评测器,它的题库仅仅是提供了各个OJ题库的一个索引,用户可以在这些题目的基础上组织比赛,然后Virtual Judge将提交的代码送到对应的OJ上去测评,再将结果返回给用户。

Topcoder采用Java applet载入平台,而不是建立于网页之上,它和codeforces都具有challenge环节,在这个环节选手可以互相查看对方代码(在一定条件下),并尝试用自己的测试用例来找出对方代码中的BUG。

\section{本文主要工作}
随着在线评测系统越来越广泛的应用到各个领域中,作者希望采用现代的互联网技术来实现一个简单的、易扩展的在线评测系统来满足当今的需求,同时借此来了解互联网新技术的使用和创新方式。

\section{论文组织架构}
本文采用了如下的结构:

第一章:引言。主要介绍本论文课题的背景,以及在线评测系统的国内外现状和本论文的主要工作,最后对论文的章节进行了一个合理的逻辑安排。

第二章:相关概念与技术。主要介绍本论文所使用到的相关技术和概念,包括互联网开发的历史和现状、常见的几种架构以、基于Git的维护流程及所使用到的开源框架。

第三章:系统需求分析。主要为本文所研究的在线评测系统进行了需求分析,在分析系统的要求后,对整个系统的功能模块进行划分,并给出了本文需要完成的功能性和非功能性需求。

第四章:系统概要设计。概要设计就是设计软件的结构,包括组成模块,模块的层次结构,模块的调用关系,每个模块的功能等等。除此之外我们还介绍了本系统的开发环境。

第五章:系统详细设计。详细设计阶段就是为每个模块完成的功能进行具体的描述,要把功能描述转变为精确的、结构化的过程描述。本章主要通过类图和包图来描述整个系统的组成和实现。