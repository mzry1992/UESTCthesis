% !Mode:: "TeX:UTF-8"

\chapter{引言}
\section{课题背景}
ACM-ICPC(Association of Computing Machinery - ACM International Collegiate Programming Contest,美国计算机协会——国际大学生程序设计竞赛)是由国际计算机界历史悠久、颇具权威性的组织ACM于1970年发起组织的年度竞赛活动,是当今国际计算机界历史悠久并得到全球公认的规模最大、水平最高的国际大学生设计竞赛。大赛旨在展示大学生创新能力、团队精神和在压力下编写程序、分析和解决问题能力,迄今已经成功举办38届。比赛涌现出的优秀学生往往被各高校和许多知名企业所看重。

ACM-ICPC以团队的形式代表各学校参赛,每队由3名队员组成\footnote{每位队员必须是在校学生,有一定的年龄限制,并且最多可以参加2次全
球总决赛和5次区域选拔赛。}。比赛期间,每队使用1台电脑需要在5个小时内使用C、C++或Java中的一种编写程序解决7到10个问题。每个问题都有一组标准的测试数据以及对应的答案,选手程序完成之后提交裁判运行,裁判机运行选手提交的程序,通过其输出于标准答案想比较来得到结果,运行的结果会判定为``AC(正确)/WA(错误)/TLE(超时)/MLE(超出内存限制)/RE(运行错误)/PE(格式错误)''中的一种并及时通知参赛队。

这项竞赛与其它竞赛最大的区别在于它采用的是机器评测的方法而不是依靠人的评价,它采用了黑盒测试\cite{beizer1995black}的思想来评判选手的程序。在黑盒测试中,测试者只知道程序的输入、输出和系统的功能,按照一定的规范设计出一系列测试案例来进行测试。在线程序评测系统(Online Judge)以此为基础,可以对多种语言的源代码进行自动编译、测试、分析及评判。除了被应用于程序设计竞赛,也有一些老师将其引入到日常的程序语言教学之中,并取得了很好的效果\cite{youfeng2009acm}\cite{guosongshan2007acm}。

电子科技大学从2005年起便开始参加这项竞赛,在最近的第38届ACM-ICPC亚洲区域赛中国大陆赛区共有成都、杭州、南京、长沙、长春5站,其中成都站的比赛由电子科技大学承办,本届比赛中,电子科技大学学子共获4金7银5铜。其中UESTC\_Aspidochelone代表队在成都站排名第二,在南京站获得亚军殊荣,顺利晋级2014年夏季在俄罗斯叶卡捷琳堡举行的世界总决赛。

\section{研究意义}
随着在线评测系统越来越广泛的应用到各个领域中,作者希望实现一个简单的、易扩展的在线评测系统来满足当今的需求,同时借此来了解互联网新技术的使用和创新方式。

\section{研究现状}
目前已经存在许多不同种类的Online Judge

\threelinetable[htbp]{onlinejudges}{\textwidth}{lll}{几个著名的评测网站}
{a & b & c\\
}{
Topcoder & & \\
Codeforces & & \\

}{
\item[a] 这里还可以添加脚注!
}
