% !Mode:: "TeX:UTF-8"

\chapter{系统测试}
软件测试是软件项目中非常重要的一部分,每次代码发生修改,我们都需要检测这次修改是否对原有的功能造成了影响。在本项目中我们主要使用了黑盒测试的方法来对各个模块进行测试。在黑盒测试过程中,我们只需关心三样东西:设置测试数据,设定预期结果,验证结果。

\section{测试方法}
\subsection{单元测试}
在计算机编程中,单元测试(又称为模块测试, Unit Testing)是针对程序模块(软件设计的最小单位)来进行正确性检验的测试工作。程序单元是应用的最小可测试部件。在过程化编程中,一个单元就是单个程序、函数、过程等;对于面向对象编程,最小单元就是方法,包括基类(超类)、抽象类、或者派生类(子类)中的方法。通常来说,我们每修改一次程序就会进行最少一次单元测试,在编写程序的过程中前后很可能要进行多次单元测试,以证实程序达到要求。

每个理想的测试案例独立于其它案例。为了测试时隔离模块,我们经常经常使用Stubbing、Mock或Fake等测试马甲程序。单元测试通常由软件开发人员编写,用于确保他们所写的代码符合软件需求和遵循开发目标。它的实施方式可以是手动的(通过纸笔),或者是做成构建自动化的一部分。

\subsection{Mock}
软件中是充满依赖关系的,例如我们会基于Service类写操作类,而Service类又是基于数据访问类(DAO)的,依次下去。单元测试的思路就是我们想在不涉及依赖关系的情况下测试代码。种测试可以在无视代码的依赖关系的情况下去测试代码的有效性。核心思想就是如果代码按设计正常工作,并且依赖关系也正常,那么他们应该会同时工作正常。

在软件开发的世界之外,``mock''一词是指模仿或者效仿。因此可以将``mock''理解为一个替身,替代者。而在软件开发中提及``mock'',通常理解为模拟对象或者Fake\footnote{mock等多代表的是对被模拟对象的抽象类,可以把fake理解为mock的实例。}。模拟对象的概念就是我们想要创建一个可以替代实际对象的对象。这个模拟对象要可以通过特定参数调用特定的方法,并且能返回预期结果。

\subsection{Stubbing}
Stubbing(桩)是指用来替换一部分功能的程序段。桩程序可以用来模拟已有程序的行为(比如一个远端机器的过程)或是对将要开发的代码的一种临时替代。因此,打桩技术在程序移植、分布式计算、通用软件开发和测试中用处很大。桩程序是一段并不执行任何实际功能的程序,只对接受的参数进行声明并返回一个合法值。这个返回值通常只是一个对于调用者来讲可接受的值即可。桩通常用在对一个已有接口的临时替换上,实际的接口程序在未来再对桩程序进行替换。

\subsection{Mockito}
我们采用了Mockito框架来进行Mock和Stubbing操作,代码保存在\textbf{src/test}目录下,文件夹有图\ref{TestDir}所示的目录结构。

\pic[htbp]{测试目录结构}{width=0.5\textwidth}{TestDir}

Spring使用\textbf{@Bean}注解和\textbf{@Autowired}来完成依赖注入操作,为了得到这些依赖的模拟对象,我们使用了Mockito框架中的mock函数。下面这段代码配置了一个UserService的模拟对象:

\noindent
\ttfamily
\hlstd{}\hllin{001\ }\hlkwc{@Bean}\\
\hllin{002\ }\hlstd{}\hlkwc{@Primary}\\
\hllin{003\ }\hlstd{}\hlkwa{public\ }\hlstd{UserService\ }\hlkwd{mockUserService}\hlstd{}\hlopt{()\ \{}\\
\hllin{004\ }\hlstd{}\hlstd{\ \ }\hlstd{}\hlkwa{return\ }\hlstd{}\hlkwd{mock}\hlstd{}\hlopt{(}\hlstd{UserService}\hlopt{.}\hlstd{}\hlkwa{class}\hlstd{}\hlopt{);}\\
\hllin{005\ }\hlstd{}\hlopt{\}}\hlstd{}\\
\mbox{}
\normalfont
\normalsize


通过这种方式得到的模拟对象可以很轻松的实现桩程序,通过Mockito框架中的when函数,我们建立一个简单的桩程序。下面这段代码配置了一个桩程序:

\noindent
\ttfamily
\hlstd{}\hllin{001\ }\hlkwd{when}\hlstd{}\hlopt{(}\hlstd{problemService}\hlopt{.}\hlstd{}\hlkwd{checkProblemExists}\hlstd{}\hlopt{(}\hlstd{}\hlnum{1000}\hlstd{}\hlopt{)).}\hlstd{}\hlkwd{thenReturn}\hlstd{}\hlopt{(}\Righttorque\\
\hllin{002\ }\hlstd{true}\hlopt{);}\hlstd{}\\
\mbox{}
\normalfont
\normalsize


这个桩程序替换了当参数为1000时的problemService的checkProblemExists方法,返回true值。

\subsection{Spring MockMVC}
Spring框架提供了MockMvc对象来模拟浏览器的真实操作。MockMvc使用\textbf{standaloneSetup}方法来由控制器生成一个模拟的浏览器对象。在成功得到该对象后,就可以使用它的\textbf{perform}方法来进行访问测试了。

\noindent
\ttfamily
\hlstd{}\hllin{001\ }\hlslc{//\ 建立一个浏览器模拟对象}\\
\hllin{002\ }\hlstd{MockMvc\ mockMvc\ }\hlopt{=\ }\hlstd{}\hlkwd{standaloneSetup}\hlstd{}\hlopt{(}\hlstd{indexController}\hlopt{)}\\
\hllin{003\ }\hlstd{}\hlstd{\ \ \ \ \ \ \ \ }\hlstd{}\hlopt{.}\hlstd{}\hlkwd{setViewResolvers}\hlstd{}\hlopt{(}\hlstd{WebMVCResource}\hlopt{.}\hlstd{}\hlkwd{viewResolver}\hlstd{}\hlopt{())}\\
\hllin{004\ }\hlstd{}\hlstd{\ \ \ \ \ \ \ \ }\hlstd{}\hlopt{.}\hlstd{}\hlkwd{setMessageConverters}\hlstd{}\hlopt{(}\hlstd{WebMVCResource}\hlopt{.}\Righttorque\\
\hllin{005\ }\hlstd{}\hlstd{\ \ \ \ \ \ \ \ }\hlstd{}\hlkwd{messageConverters}\hlstd{}\hlopt{())}\\
\hllin{006\ }\hlstd{}\hlstd{\ \ \ \ \ \ \ \ }\hlstd{}\hlopt{.}\hlstd{}\hlkwd{build}\hlstd{}\hlopt{();}\\
\hllin{007\ }\hlstd{}\\
\hllin{008\ }\hlslc{//\ 模拟Get操作,访问/地址}\\
\hllin{009\ }\hlstd{}\hlslc{//\ 期望结果为:}\\
\hllin{010\ }\hlstd{}\hlslc{//}\hlstd{\ \ \ \ }\hlslc{返回状态为OK}\\
\hllin{011\ }\hlstd{}\hlslc{//}\hlstd{\ \ \ \ }\hlslc{视图名字是index/index}\\
\hllin{012\ }\hlstd{}\hlslc{//}\hlstd{\ \ \ \ }\hlslc{且模型中有期望的消息}\\
\hllin{013\ }\hlstd{mockMvc}\hlopt{.}\hlstd{}\hlkwd{perform}\hlstd{}\hlopt{(}\hlstd{}\hlkwd{get}\hlstd{}\hlopt{(}\hlstd{}\hlstr{"/"}\hlstd{}\hlopt{))}\\
\hllin{014\ }\hlstd{}\hlstd{\ \ \ \ \ \ \ \ }\hlstd{}\hlopt{.}\hlstd{}\hlkwd{andExpect}\hlstd{}\hlopt{(}\hlstd{}\hlkwd{status}\hlstd{}\hlopt{().}\hlstd{}\hlkwd{isOk}\hlstd{}\hlopt{())}\\
\hllin{015\ }\hlstd{}\hlstd{\ \ \ \ \ \ \ \ }\hlstd{}\hlopt{.}\hlstd{}\hlkwd{andExpect}\hlstd{}\hlopt{(}\hlstd{}\hlkwd{view}\hlstd{}\hlopt{().}\hlstd{}\hlkwd{name}\hlstd{}\hlopt{(}\hlstd{}\hlstr{"index/index"}\hlstd{}\hlopt{))}\\
\hllin{016\ }\hlstd{}\hlstd{\ \ \ \ \ \ \ \ }\hlstd{}\hlopt{.}\hlstd{}\hlkwd{andExpect}\hlstd{}\hlopt{(}\hlstd{}\hlkwd{model}\hlstd{}\hlopt{().}\hlstd{}\hlkwd{attribute}\hlstd{}\hlopt{(}\hlstd{}\hlstr{"message"}\hlstd{}\hlopt{,\ }\hlstd{}\hlstr{"home\ page.}\Righttorque\\
\hllin{017\ }\hlstr{}\hlstd{\ \ \ \ \ \ \ \ }\hlstr{"}\hlstd{}\hlopt{));}\hlstd{}\\
\mbox{}
\normalfont
\normalsize


\section{测试内容}
\subsection{数据库模块测试}
这里对Condition模块进行了测试。我们通过Condition类来生成HQL查询语句中的条件语句,它在服务器运行过程中是最常使用到的模块之一,所以对它的测试要求比较高,这部分的测试一共有12组,见表\ref{conditionTest}。

%\threelinetable[htbp]{conditionTest}{\textwidth}{ll}{Condition模块测试用例}
\longthreelinetable{conditionTest}{Condition模块测试用例表}{2}{ll}
{测试描述 & 期望结果\\
}{
条件为空 & 空字符串\\
简单的样例 & \textsf{where (id$=$'1')}\\
与条件测试 & \textsf{where (id$>=$'1' and id$<=$'5')}\\
或条件测试 & \textsf{where (id$>=$'1' or id$<=$'5')}\\
嵌套测试 & \textsf{where ((id$>=$'1' or id$<=$'5') and (price$>$'10' or price$<=$'20'))}\\
非空条件测试 & \textsf{where ((userId is not null))}\\
空条件测试 & \textsf{where ((userId is null))}\\
嵌套测试 & \textsf{where ((userId is not null) and (departmentId is null))}\\
排序测试 & \textsf{ order by userId desc}\\
条件非空排序测试 & \textsf{where (userId$>=$'1') order by userId desc}\\
多关键字排序测试 & \textsf{where (userId$>=$'1') order by departmentId asc,userId desc}\\
组合测试 & \textsf{where ((userId$>=$'1' or userId$<=$'5') and ((departmentId is not null) or }\\
& \textsf{userName like '\%user\%')) order by departmentId asc,userId desc}\\
}{
}

\subsection{数据库集成测试}
数据库集成测试的作用是检测数据库各个模块之间是否工作正常。我们构建了一个名叫\textsf{uestcojtest}的测试数据库,这个数据库里面的每个表中都包含了若干数据,用以进行数据库集成测试。这里面包含了许多子测试,见表\ref{databaseITTest}。

\longthreelinetable{databaseITTest}{数据库集成测试用例表}{3}{lll}
{测试函数 & 测试内容 & 期望结果\\
}{
 & \multicolumn{2}{c}{基本测试}\\
 \cmidrule[0.05em]{2-3}
testFetchDataSource & 当前使用的数据库 & \textsf{uestcojtest}\\
testDataBaseConnection & 数据库连接情况 & 正常\\
testHQLQuery & HQL查询 & 正常\\

 & \multicolumn{2}{c}{Condition测试}\\
 \cmidrule[0.05em]{2-3}
testCondition\_emptyEntrySet & 查询功能 & 正常\\
testCondition\_emptyEntrySetWithDescId & 查询功能(按ID逆序排列)& 正常\\
testCondition\_count\_emptyCondition & 个数查询功能 & 正常\\
testCondition\_count\_withDepartmentId & 个数查询功能(带条件) & 正常\\

 & \multicolumn{2}{c}{查找功能测试(返回实体)}\\
 \cmidrule[0.05em]{2-3}
testDAO\_getEntityByUnique & 依照主键查找 & 正常\\
testDAO\_getEntityByUnique\_notUniqueField & 依照非主键查找 & 抛出异常\\

 & \multicolumn{2}{c}{查找功能测试(返回DTO)}\\
 \cmidrule[0.05em]{2-3}
testDAO\_findAllByBuilder & 查找操作 & 正常\\
testDAO\_findAllByBuilder\_withPageInfo & 查找操作(带目录限制) & 正常\\
testDAO\_getDTOByUniqueField\_null & 查找失败 & 返回null\\
testDAO\_getDTOByUniqueField\_intType & 查找操作(条件为数字) & 正常\\
testDAO\_getDTOByUniqueField\_stringType & 查找操作(条件为字符串) & 正常\\
testDAO\_getDTOByUniqueField\_failed & 依照非主键查找 & 抛出异常\\

 & \multicolumn{2}{c}{更新功能测试}\\
 \cmidrule[0.05em]{2-3}
testSQLUpdate & 多行更新 & 正常\\

 & \multicolumn{2}{c}{DTO功能测试}\\
 \cmidrule[0.05em]{2-3}
testUserDTO & Builder功能 & 正常\\

 & \multicolumn{2}{c}{ContestProblemDAO测试}\\
 \cmidrule[0.05em]{2-3}
testAddContestProblem & 添加比赛题目功能 & 正常\\

 & \multicolumn{2}{c}{Problem数据库测试}\\
 \cmidrule[0.05em]{2-3}
testStartIdAndEndId & 按ID区域查找 & 正常\\
testStartIdAndEndId\_invalidParameter & 按ID区域查找(参数错误) & 空列表\\
testIsSpjQuery\_notSpj & 按SPJ状态查找(非SPJ) & 正常\\
testIsSpjQuery\_spj & 按SPJ状态查找(是SPJ) & 正常\\
testProblemCondition\_emptyTitle & 按标题查找(空标题) & 正常\\
testAddProblem & 添加题目 & 正常\\

 & \multicolumn{2}{c}{Status数据库测试}\\
 \cmidrule[0.05em]{2-3}
testStatusDAO\_withDistinctProblem & 查找符合条件的状态的题目ID & 正常\\

 & \multicolumn{2}{c}{Tag数据库测试}\\
 \cmidrule[0.05em]{2-3}
testQuery\_fetchAllTags & 查找所有的tag & 正常\\
testCount & 查找所有的tag的个数 & 正常\\

 & \multicolumn{2}{c}{User数据库测试}\\
 \cmidrule[0.05em]{2-3}
 testQuery\_byName & 按姓名查找用户 & 正常\\
 testQuery\_byDepartmentId & 按学院查找用户 & 正常\\
 testUserCondition\_byStartIdAndEndId & 按ID查找用户 & 正常\\

 & \multicolumn{2}{c}{UserSerialKey数据库测试}\\
 \cmidrule[0.05em]{2-3}
 testFindUserSerialKeyByUserName & 按用户姓名查找激活码 & 正常\\
}{
}

\subsection{服务集成测试}
前面我们已经对数据库进行了集成测试,在这些测试都通过后,我们可以假设数据库模块已经是正常的了,然后在服务中用DAO的模拟对象对服务模块进行测试。这里面包含了许多子测试,由于篇幅限制我们只给出其中两个服务的测试用例,见表\ref{serviceITTest}。

\longthreelinetable{serviceITTest}{服务集成测试用例表}{3}{lll}
{测试函数 & 测试内容 & 期望结果\\
}{
 & \multicolumn{2}{c}{ProblemService测试}\\
 \cmidrule[0.05em]{2-3}
 testGetProblemDTOByProblemId & 按题目ID查询ProblemDTO & 正常\\
 testCount & 题目数量查询 & 正常\\
 testUpdateProblem & 题目更新 & 正常\\
 testUpdateProblem\_problemNotFound & 题目更新(更新不存在的题目) & 抛出异常\\
 testUpdateProblem\_problemFoundWithNullId & 题目更新(题目ID错误) & 抛出异常\\
 testCreateNewProblem & 新建题目 & 正常\\
 testGetProblemListDTOList & 获取题目列表 & 正常\\
 testGetAllVisibleProblemIds & 获取可见题目ID列表 & 正常\\

 & \multicolumn{2}{c}{UserService测试}\\
 \cmidrule[0.05em]{2-3}
testGetUserByUserId & 按用户ID查询用户 & 正常\\
testGetUserByUserName & 按用户名查询用户 & 正常\\
testGetUserByUserEmail & 按用户邮箱查询用户 & 正常\\
testUpdateUser & 更新用户 & 正常\\
testCount\_emptyCondition & 用户数量查询(条件为空)&正常\\
testCount\_byStartId & 用户数量查询(ID大于等于2)&正常\\
testCount\_byEndId & 用户数量查询(ID小于等于2)&正常\\
testCount\_byStartIdAndEndId & 用户数量查询(ID在2到10之间)&正常\\
testCount\_byStartIdAndEndId\_empty & 用户数量查询(不存在ID区间)&正常\\
testCount\_byDepartmentId & 用户数量查询(按学院查询)&正常\\
testCount\_bySchool & 用户数量查询(按学校查询)&正常\\
testCount\_bySchool\_empty & 用户数量查询(不存在的学校)&正常\\
testCount\_byUserName & 用户数量查询(按用户名查询)&正常\\
testCount\_byType & 用户数量查询(按类型查询)&正常\\
}{
}

\subsection{实用工具测试}
系统中我们用到了许多实用工具,例如数据解压工具、数组工具等,这些工具我们也作出了相应的测试。测试用例见表\ref{utilITTest}

\longthreelinetable{utilITTest}{实用工具测试用例表}{3}{lll}
{测试函数 & 测试内容 & 期望结果\\
}{
 & \multicolumn{2}{c}{题目数据解压工具测试}\\
 \cmidrule[0.05em]{2-3}
 testCheck\_withoutSpjFile\_oneCase & 一组数据,无SPJ & 解压正常\\
 testCheck\_withoutSpjFile\_moreCases & 多组数据,无SPJ & 解压正常\\
 testCheck\_withSpjFile\_oneCase & 一组数据,有SPJ & 解压正常\\
 testCheck\_withSpjFile\_moreCases & 多组数据,有SPJ & 解压正常\\
 testCheck\_withInvalidDataName & 无效的数据 & 解压失败\\
 testCheck\_withDirectory & 包含无效目录 & 解压失败\\
 testCheck\_withNullDirectory & 空压缩包 & 解压失败\\
 testCheck\_invalidDataDirectory & 非法数据文件夹 & 解压失败\\
 testCheck\_notSameInputsAndOutputs & 无对应的输出文件 & 解压失败\\
 testCheck\_missMatchInputAndOutput & 输入输出文件不匹配 & 解压失败\\

 & \multicolumn{2}{c}{数组工具测试}\\
 \cmidrule[0.05em]{2-3}
 testParseIntArray & 解析整数数组 & 正常\\

 & \multicolumn{2}{c}{文件比较工具测试}\\
 \cmidrule[0.05em]{2-3}
 testSame & 完全相等 & 相等\\
 testSame\_deletingWhiteSpace & 有多余空格 & 相等\\
 testSame\_endingSpaces & 末尾空格 & 相等\\
 testSame\_tabSpaces & 多余tab & 相等\\
 testDifferent\_endingSpaces & 不同的字符串 & 不等\\
 testDifferent\_specialCharacter & 不同字符串,特殊字符 & 不等\\
}{
}

\subsection{控制器测试}
在服务测试通过的情况下,我们可以在控制器中设置服务的模拟对象来进行控制器测试。这里面包含了许多子测试,由于篇幅限制我们只给出其中三个控制器的测试用例,见表\ref{controllerTest}。

\longthreelinetable{controllerTest}{控制器测试用例表}{3}{lll}
{测试函数 & 测试内容 & 期望结果\\
}{
 & \multicolumn{2}{c}{IndexController测试}\\
 \cmidrule[0.05em]{2-3}
 testVisitIndex & 主页访问 & 正常\\

 & \multicolumn{2}{c}{ProblemController测试}\\
 \cmidrule[0.05em]{2-3}
 testShow\_successful & 题目页面(访问成功) & 正常\\
 testShow\_problemNotFound & 题目页面(不存在的题目) & 跳转到404页面\\
 testList & 题目列表页面 & 正常\\

 & \multicolumn{2}{c}{UserController测试}\\
 \cmidrule[0.05em]{2-3}
testLogin\_successful & 登陆(成功) & 正常\\
testLogin\_invalidUserName\_null & 登陆(用户名为空) & 验证失败\\
testLogin\_invalidUserName\_tooShort & 登陆(用户名太短) & 验证失败\\
testLogin\_invalidUserName\_tooLong & 登陆(用户名太长) & 验证失败\\
testLogin\_invalidUserName\_invalid & 登陆(用户名非法) & 验证失败\\
testLogin\_invalidPassword\_null & 登陆(密码为空) & 验证失败\\
testLogin\_invalidPassword\_tooShort & 登陆(密码太短) & 验证失败\\
testLogin\_invalidPassword\_tooLong & 登陆(密码太长) & 验证失败\\
testLogin\_failed\_wrongUserNameOrPassword & 登陆(密码错误) & 验证失败\\
testLogin\_failed\_bothUserNameAndPassword\_null & 登陆(用户名密码都为空) & 验证失败\\
testLogin\_failed\_serviceError & 登陆(系统错误) & 返回错误信息\\
testLogout\_successful & 登出(成功) & 正常\\
testRegister\_successfully & 注册(成功) & 正常\\
testRegister\_failed\_userName\_null & 注册(用户名为空) & 验证失败\\
testRegister\_failed\_userName\_whiteSpaces & 注册(用户名有空格) & 验证失败\\
testRegister\_failed\_userName\_tooShort & 注册(用户名太短) & 验证失败\\
testRegister\_failed\_userName\_tooLong & 注册(用户名太长) & 验证失败\\
testRegister\_failed\_userName\_invalid & 注册(用户名非法) & 验证失败\\
testRegister\_failed\_password\_null & 注册(密码为空) & 验证失败\\
testRegister\_failed\_password\_tooShort & 注册(密码太短) & 验证失败\\
testRegister\_failed\_password\_tooLong & 注册(密码太长) & 验证失败\\
testRegister\_failed\_passwordRepeat\_null & 注册(密码确认为空) & 验证失败\\
testRegister\_failed\_passwordRepeat\_tooShort & 注册(密码确认太短) & 验证失败\\
testRegister\_failed\_passwordRepeat\_tooLong & 注册(密码确认太长) & 验证失败\\
testRegister\_failed\_passwordRepeat\_different & 注册(密码确认与密码不同) & 验证失败\\
testRegister\_failed\_nickName\_null & 注册(昵称为空) & 验证失败\\
testRegister\_failed\_nickName\_whiteSpaces & 注册(昵称有空格) & 验证失败\\
testRegister\_failed\_nickName\_tooShort & 注册(昵称太短) & 验证失败\\
testRegister\_failed\_nickName\_tooLong & 注册(昵称太长) & 验证失败\\
testRegister\_failed\_nickName\_invalid & 注册(昵称非法) & 验证失败\\
testRegister\_failed\_email\_invalid & 注册(邮箱错误) & 验证失败\\
testRegister\_failed\_school\_tooShort & 注册(学校名太短) & 验证失败\\
testRegister\_failed\_school\_tooLong & 注册(学校名太长) & 验证失败\\
testRegister\_failed\_departmentId\_null & 注册(学院为空) & 验证失败\\
testRegister\_failed\_departmentNotFound & 注册(学院不存在) & 验证失败\\
testRegister\_failed\_studentId\_tooShort & 注册(学号太短) & 验证失败\\
testRegister\_failed\_studentId\_tooLong & 注册(学号太长) & 验证失败\\
testRegister\_failed\_usedUserName & 注册(用户名已被占用) & 验证失败\\
testRegister\_failed\_usedEmail & 注册(邮箱已被占用) & 验证失败\\
testUser\_register\_login\_logout & 用户注册、登陆、登出操作 & 正常\\
}{
}

控制器是直接和用户打交道的函数,我们选择了非常多的数据来检验用户提交表单的各种情况,防止用户的恶意提交对数据库造成破坏。

\section{测试结果}
在\textbf{trunk}目录下运行\textbf{mvn test}即可进行自动化测试,下面是测试的运行结果:
\begin{verbatim}
-------------------------------------------------------
 T E S T S
-------------------------------------------------------
Running TestSuite
PASSED: UserServiceTest
PASSED: ProblemServiceTest
PASSED: ConditionTest
PASSED: ArrayUtilTest
PASSED: CompareSkipSpacesTest
PASSED: ZipDataCheckerTest
PASSED: IndexControllerTest
PASSED: UserControllerTest
PASSED: ProblemControllerTest
PASSED: Database integration tests
PASSED: Service integration tests
PASSED: Utility integration tests
Tests run: 183, Failures: 0, Errors: 0, Skipped: 0
Time elapsed: 10.53 sec - in TestSuite

Results :

Tests run: 183, Failures: 0, Errors: 0, Skipped: 0
\end{verbatim}
可以看到所有测试用例全部通过测试。