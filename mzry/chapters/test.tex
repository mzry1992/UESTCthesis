% !Mode:: "TeX:UTF-8"

\chapter{系统测试}
软件测试是软件项目中非常重要的一部分,每次代码发生修改,我们都需要检测这次修改是否对原有的功能造成了影响。在本项目中我们主要使用了黑盒测试的方法来对各个模块进行测试。在黑盒测试过程中,我们只需关心三样东西:设置测试数据,设定预期结果,验证结果。

\section{测试方法}
\subsection{单元测试}
在计算机编程中,单元测试(又称为模块测试, Unit Testing)是针对程序模块(软件设计的最小单位)来进行正确性检验的测试工作。程序单元是应用的最小可测试部件。在过程化编程中,一个单元就是单个程序、函数、过程等;对于面向对象编程,最小单元就是方法,包括基类(超类)、抽象类、或者派生类(子类)中的方法。通常来说,我们每修改一次程序就会进行最少一次单元测试,在编写程序的过程中前后很可能要进行多次单元测试,以证实程序达到要求。

每个理想的测试案例独立于其它案例。为了测试时隔离模块,我们经常经常使用Stubbing、Mock或Fake等测试马甲程序。单元测试通常由软件开发人员编写,用于确保他们所写的代码符合软件需求和遵循开发目标。它的实施方式可以是非常手动的(通过纸笔),或者是做成构建自动化的一部分。

\subsection{Mock}
软件中是充满依赖关系的,例如我们会基于Service类写操作类,而Service类又是基于数据访问类(DAO)的,依次下去。单元测试的思路就是我们想在不涉及依赖关系的情况下测试代码。种测试可以在无视代码的依赖关系的情况下去测试代码的有效性。核心思想就是如果代码按设计正常工作,并且依赖关系也正常,那么他们应该会同时工作正常。

在软件开发的世界之外,``mock''一词是指模仿或者效仿。因此可以将``mock''理解为一个替身,替代者。而在软件开发中提及``mock'',通常理解为模拟对象或者Fake\footnote{mock等多代表的是对被模拟对象的抽象类,可以把fake理解为mock的实例。}。模拟对象的概念就是我们想要创建一个可以替代实际对象的对象。这个模拟对象要可以通过特定参数调用特定的方法,并且能返回预期结果。

\subsection{Stubbing}
Stubbing(桩)是指用来替换一部分功能的程序段。桩程序可以用来模拟已有程序的行为(比如一个远端机器的过程)或是对将要开发的代码的一种临时替代。因此,打桩技术在程序移植、分布式计算、通用软件开发和测试中用处很大。桩程序是一段并不执行任何实际功能的程序,只对接受的参数进行声明并返回一个合法值。这个返回值通常只是一个对于调用者来讲可接受的值即可。桩通常用在对一个已有接口的临时替换上,实际的接口程序在未来再对桩程序进行替换。

\section{测试内容}
