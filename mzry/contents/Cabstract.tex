% !Mode:: "TeX:UTF-8"

\begin{Cabstract}{程序设计竞赛}{黑盒测试}{在线程序评测系统}{MVC架构}{Web应用}
ACM国际大学生程序设计竞赛(英语:ACM International Collegiate Programming Contest, ICPC)是由美国计算机协会(ACM)主办的,一项旨在展示大学生 创新能力、团队精神和在压力下编写程序、分析和解决问题能力的年度竞赛。经过30多年的发展,ACM国际大学生程序设计竞赛已经发展成为最具影响力的大学生计算机竞赛。

程序设计竞赛采用了黑盒测试的思想来评判选手的程序。在黑盒测试中,测试者只知道程序的输入、输出和系统的功能,按照一定的规范设计出一系列测试案例来进行测试。在线程序评测系统(Online Judge)以此为基础,可以对多种语言的源代码进行自动编译、测试、分析及评判。除了被应用于程序设计竞赛,也有一些老师将其引入到日常的程序语言教学之中,并取得了很好的效果。

在本文中,作者首先通过介绍了目前著名的几个在线评测系统来说明目前该类平台的发展趋势,并阐述了MVC架构的特点以及如何应用到web应用中去,随后设计了一个简单的分布式在线程序评测系统,并描述了它的整体架构和实现。
\end{Cabstract}
